\documentclass[a4paper,12pt]{article}
\usepackage{cmap}
\usepackage[utf8]{inputenc}
\usepackage[russian]{babel}

\usepackage{amsmath}
\usepackage[left=2cm,right=2cm,top=3cm,bottom=2cm]{geometry}
\usepackage[dvips]{graphicx}
\graphicspath{{\noiseimages}}
\usepackage{graphicx}

\headheight16pt

\usepackage{indentfirst}

\usepackage{fancyhdr}
\pagestyle{fancy}
\fancyhead{}
\fancyhead[LO]{Межрегиональная физическая олимпиада 2014---2015. Заочный тур}
\fancyhead[RO]{Стр.~\thepage~из 3}
\fancyfoot{}

\newcommand\hr[1]{\left({#1}\right)}
\newcommand\un[1]{\,\emph{#1}}

\def\thesection{\arabic{section}.}
\def\thesubsection{\arabic{section}.\arabic{subsection}.}

\begin{document}

\section{Теоретические задачи}

\subsection{Гром и молния}
Однажды знаменитый маг и волшебник Дмитрий К., разгуливая по лесу, наблюдал вспышку молнии,
но не услышал грома. Есть ли в этом магия, или такое может быть в повседневной жизни?

\subsection{Алхимия}
Алхимик Всеволод~XV просыпал порошок меди и стеклянный песок на пол, а потом ещё
и~наступил на этот бардак. В результате этот алхимический мусор был перемешан,
и Всеволод~XV задумался, как быстро отделить одно от другого, не используя магию.
А как бы это сделал ты?

\subsection{Пара экспериментов}
Начинающий разработчик философского камня Егор опустил в стакан с жидкостью кусочек
льда с полостью внутри и замерил уровень жидкости. Потом Егор магически заполнил
полость холодным свинцом и снова замерил уровень жидкости. При каких условиях
внешней среды (особенности жидкости, атмосферы, эфирного ветра), постоянных
для обоих тестов, значения будут отличаться?

\subsection{Какой идти дорогой?}
Миша в первый раз приехал на Летнюю экологическую школу преподавателем физики.
От лагеря физиков до столовой есть два маршрута: один всё время идёт в гору
под одним и тем же углом; другой на половине пути ровный, а вторую половину
дистанции занимает подъём. По карте маршруты имеют одинаковую длину.
По какому пути Мише лучше ехать?

\subsection{Качели}
Больше всего на свете Маша любит физику и качели! Как-то раз, качаясь на качелях,
Маша заметила, что качели скрипят только в одном направлении на высокой ноте,
а обратно на низкой. Немного поразмыслив, девочка нашла объяснение этому факту.
Как ты думаешь, какое?

\section{Экспериментальные задачи}

\subsection{Цена слова}
Как известно, то, сколько марок клеить на почтовый конверт, зависит не только от того,
куда письмо необходимо отправить, но и от веса самого письма. А ведь каждое слово имеет
свой вес! Попробуй экспериментально определить, сколько весит написанное простым
карандашом на бумаге слово <<олимпиада>>. Для определённости будем считать, что
оно написано поперёк листа A4 крупными буквами (19\emph{см} в длину и 4~\emph{см} в~высоту).

\subsection{Плотность раствора}
Начинающий естествоиспытатель Лямбда решил выяснить, как зависит плотность
водного раствора какого-либо вещества, от концентрации этого вещества
в растворе (например, соли, уж её-то везде легко достать). Присоединись
к исследованию Лямбды и попробуй самостоятельно собрать необходимый
для этого прибор и провести соответствующие измерения.

\subsection{Волчок}
Если раскручивать разноцветный «волчок», то при некоторой скорости его цвета будут сливаться.
Попробуй экспериментально определить, при какой скорости это произойдет, если <<волчок>>:\\
\begin{enumerate}
\item[а] двухцветный
\item[б] восьмицветный
\item[в] двадцитичетырёхцветный
\end{enumerate}

\subsecton{Конденсаторы}
Для построения новой модели летающей тарелки межгалактического типа гуманоиду Дане
совершенно необходимо много конденсаторов! Но вот беда: на его родной планете их нет.
Помоги гуманоиду Дане сделать из подручных материалов конденсатор ёмкостью 100~\emph{pF}.
Не забудь, что после этого необходимо (в  целях проверки) измерить эту ёмкость,
обосновать способ измерения и оценить его точность.

\subsection{Магниты}
Глеб очень любит магниты! Он собирает любые магниты, какие только может обнаружить!
А как их обнаружить, если не по создаваемому ими магнитному полю? Постарайся придумать
и проверить на практике любой способ обнаружения магнитного поля
на максимальном расстоянии от его источника.

%\medskip
%\centerline{\includegraphics[width=0.5\linewidth]{VCP21}}
%\medskip
%\centerline{\small Рис.~1}

\end{document}
