Большая просьба: \textbf{пишите разборчиво, крупно и ярко выделяйте номера задач}.
При оформлении экспериментальных задач крайне желательно предоставить
\ctxitems{
\item рисунок или фотографию установки
\item схему эксперимента (последовательность действий)
\item результаты эксперимента (лучше в виде таблицы)
}

Каким бы способом Вы ни оформляли работу, лучше всего отослать её на проверку по~электронной
почте \href{mailto:\olympmail}{\olympmail}
или через сайт \href{http://fizlesh.ru/contest/send}{fizlesh.ru/contest/send}.
Если Вы оформляете её на бумаге, отсканируйте или сфотографируйте работу
(пожалуйста, для обеспечения читаемости, не пользуйтесь для этого камерами на телефонах).
В~любом случае проверьте электронный вариант вашей работы на читаемость!
Решение, которое мы не сможем по той или иной причине разобрать, будет приравнено к его отсутствию.

\textbf{Обратите внимание}, что электронная почта и форма загрузки сайта могут не пропускать
документы большого объёма (\textbf{более 14-15 мегабайт}). В случае, если Ваша работа в электронном
виде превысила указанный объём, разбейте её на части или воспользуйтесь файлообменными сервисами
\mbox{Яндекс.Диск} (или \mbox{Файлы.Mail.RU}, или любым другим), получите ссылку на загруженный файл
и~пришлите её нам (через форму на сайте или по электронной почте \href{mailto:\olympmail}{\olympmail}).

Если у Вас возникли проблемы с~отправкой работы, пожалуйста, напишите по адресу
\href{mailto:demarin@mail.ru}{demarin@mail.ru} или позвоните по телефону
\textbf{+7(910)473-69-06} (Дмитрий), и мы постараемся Вам помочь.

Даже если у Вас нет возможности перевести работу в электронный вид, отправьте нам уведомление
о~том, что выслали её почтой. Уведомления высылайте
на электронный адрес \href{mailto:\olympmail}{\olympmail}.
Отправить работу почтой можно на адрес \emph{121357, г.~Москва, ул.~Кременчугская, д.~13,
ГБОУ школа-интернат <<Интеллектуал>>, Шувалову В.\,Ю.}
