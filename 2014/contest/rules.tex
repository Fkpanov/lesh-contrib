\documentclass[a4paper,12pt]{article}
\usepackage{cmap}
\usepackage{hyperref}
\usepackage[utf8]{inputenc}
\usepackage[russian]{babel}

\usepackage{amsmath}
\usepackage[left=2cm,right=2cm,top=3cm,bottom=2cm]{geometry}

\headheight16pt

\usepackage{indentfirst}

\usepackage{fancyhdr}
\pagestyle{fancy}
\fancyhead{}
\fancyhead[LO]{Межрегиональная физическая олимпиада 2013---2014. Правила}
\fancyhead[RO]{Стр.~\thepage~из 2}
\fancyfoot{}

\newcommand\hr[1]{\left({#1}\right)}
\newcommand\un[1]{\,\emph{#1}}

\def\thesection{\arabic{section}.}
\def\thesubsection{\arabic{section}.\arabic{subsection}.}

\title{\bf Межрегиональная физическая олимпиада\\Правила заочного тура}
\author{ГБОУ <<Интеллектуал>> г. Москвы --- Физическое Отделение ЛЭШ\\
fizlesh.ru/contest}

\begin{document}
\maketitle
\thispagestyle{empty}

Межрегиональная олимпиада состоит из заочного и очного туров. Заочный тур проводится
\textbf{с~20~октября 2013 года} до \textbf{20~ноября 2013 года} включительно.
До этого дня (не обязательно в последний день, лучше раньше) необходимо выслать решения предложенных задач.
Условия задач можно увидеть также на сайте \href{http://fizlesh.ru/contest}{fizlesh.ru/contest}.
Там же можно найти более подробную информацию об олимпиаде и её организаторах.

\bigskip

Работу можно оформлять как в виде электронного документа, так и в письменном виде, на двойных тетрадных листах.
В работе должны присутствовать:
\begin{enumerate}
\setlength{\itemsep}{-3pt}
\item Титульный лист (на нём не должно быть решений задач)
\item Решения задач теоретической части
\item Решения задач экспериментальной части
\end{enumerate}

Титульный лист заполняется в соответствии со следующими пунктами:
\begin{enumerate}
\setlength{\itemsep}{-3pt}
\item Фамилия, имя, отчество участника олимпиады (полностью, печатными буквами)
\item Фамилии, имена, отчества родителей (полностью)
\item Школа, класс
\item Домашний адрес полностью, с индексом, названием населённого пункта и региона
\item Контактный телефон
\item Действующий адрес электронной почты (крайне желателен для оповещения о приглашении на второй тур)
\item Название детского объединения (кружок, клуб) по физике, которое посещаете,\\
Ф.И.О. руководителей (полностью, печатными буквами)
\item Фамилия, имя, отчество учителя физики (полностью, печатными буквами)
\end{enumerate}


\bigskip

Решать все задачи вовсе не обязательно. Лучше максимально полно ответить на вопросы задач,
рассмотреть интересные случаи. В некоторых задачах возможно несколько решений, базирующихся
на разных идеях. За неверные версии оценка не снижается.


\bigskip


Возможно использование литературы и~других источников информации (в том числе сети Интернет).
Однако работа выполняется индивидуально, пользоваться помощью сверстников и учителей не разрешается.
Обращаем Ваше внимание, что в случае обнаружения признаков списывания друг у друга,
иных форм <<коллективного творчества>> и~других нарушений Правил проведения Олимпиады,
Оргкомитет оставляет за собой право дисквалифицировать участников, прошедших по сумме баллов во второй тур.


\bigskip


Большая просьба: пишите разборчиво, крупно и ярко выделяйте номера задач.
При оформлении экспериментальных задач крайне желательно предоставить
\begin{itemize}
\setlength{\itemsep}{-3pt}
\item рисунок или фотографию установки
\item схему эксперимента (последовательность действий)
\item результаты эксперимента (лучше в виде таблицы)
\end{itemize}

Каким бы способом Вы ни оформляли работу, лучше всего отослать её на проверку по~электронной
почте или через сайт (\href{http://fizlesh.ru/contest/send}{fizlesh.ru/contest/send}.
Если Вы оформляете её на бумаге, отсканируйте или сфотографируйте работу
(пожалуйста, для обеспечения читаемости, не пользуйтесь для этого камерами на телефонах).
\textbf{Важно}: в любом случае проверьте электронный вариант вашей работы на читаемость!
Решение, которое мы не сможем по той или иной причине разобрать, будет приравнено к его отсутствию.

Даже если у Вас нет возможности перевести работу в электронный вид, отправьте нам уведомление
о~том, что выслали её почтой. Свои работы или уведомления об отправке Вы можете высылать
на электронный адрес \href{mailto:contest@fizlesh.ru}{contest@fizlesh.ru}
или через форму на сайте \href{http://fizlesh.ru/contest}{fizlesh.ru/contest}.
Отправить работу почтой можно на адрес \emph{121357, г.~Москва, ул.~Кременчугская, д.~13, ГБОУ школа-интернат
<<Интеллектуал>>, Шувалову В.\,Ю.}


\bigskip


10 января 2014~г. закончится проверка работ и определение победителей первого тура.
Победители будут приглашены на второй тур олимпиады~--- Всероссийскую Весеннюю Многопредметную Школу
в г.~Пущино. Списки приглашённых появятся на сайте\\
\centerline{\href{http://fizlesh.ru/contest}{fizlesh.ru/contest}.}
\textbf{Важно}: Будьте готовы достаточно оперативно ответить на письмо, которое может
прийти к Вам на e-mail в этих числах.

\bigskip

На Школе будет прочтено много интересных курсов и факультативов, проведены различные экскурсии.
Участники, победившие в олимпиаде по итогам второго тура, будут награждены призами и грамотами.
Самые заинтересованные и успешные участники будут приглашены на Физическое отделение Летней Экологической Школы.

\bigskip

Удачи!

\end{document}
