\documentclass[12pt]{article}
\usepackage[utf8]{inputenc}
\usepackage[english,russian]{babel}
\usepackage{konduit01-utf8}

\textheight=210mm

\newcommand{\pstar}[1][]{\refstepcounter{pnum}{\immediate\write\tempfile{\arabic{znum}.\arabic{pnum}.\pstyle.\if\relax\detokenize{#1}\relax\zprev\else#1\fi}} {\bf\thepnum}$^{\mbox{#1}*}${\bf{)}$\;$}}

\begin{document}

\name[Базис, размерность]{2}{}{}

{\defin Пусть $L$ --- линейное пространство над числами $F$. Вектор вида $x=\alpha_1 u_1+\alpha_2 u_2+...\alpha_n u_n$, где $\alpha_i \in F$, $u_i \in L $, называется {\it линейной комбинацией} векторов $u_1,...,u_n$ (если $n=0$, то полагаем $x=0$). Если все векторы $u_i$ различны, а коэффициенты перед ними --- ненулевые, то такая линейная комбинация называется {\it неприводимой}.}

\z Докажите, что множество всех линейных комбинаций векторов из множества $U\subset L$ является линейным подпространством пространства $L$. Это подпространство называется {\it линейной оболочкой} и обозначается $\langle U\rangle$.

{\defin Векторы множества $U$ называются линейно независимыми, если никакая неприводимая линейная комбинация ненулевого числа элементов $U$ не равна нулю.
}

\z Являются ли линейно независимыми векторы следующих множеств:
	\p $\{(1, -1, 0), (-1, 0, 1), (0, 1, -1)\} \subset \R^3$
	\p $\{(1, 1, 0), (1, 0, 1), (0, 1, 1)\} \subset \R^3$
	\p $\{(1,0,1), (0,0,0), (1,1,0)\} \subset \R^3$
	\p $\{(1,2), (1,-1)\} \subset \R^2$ \\
	\p $\{(1,3), (-4,-12)\} \subset \R^2$

\z Векторы множества $U$ являются линейно зависимыми,
если и только если один из них есть линейная комбинация других.

\z Пусть из трех векторов $e_1$, $e_2$, $e_3$ любые два линейно независимы. Могут ли все три вектора быть линейно зависимыми?



{\defin Базисом линейного пространства $L$ называется
множество линейно независимых векторов из $L$, линейная оболочка
которого совпадает с $L$.}

\z Множество векторов является базисом $L$, если и только
если любой вектор из $L$ выражается в виде неприводимой линейной
комбинации векторов этого множества единственным (с точностью до
перестановки слагаемых) образом.

\z Если один из двух базисов есть подмножество другого, то
эти два базиса совпадают.

\z Пусть $X$, $X'$ --- базисы пространства $L$. Доказать, что
для любого вектора $x \in X \setminus X'$
найдется такой вектор $y \in X' \setminus X$ , что
$X \cup \{y\}\setminus\{x\}$ есть базис $L$.


\defin Линейное пространство $L$ называется {\it конечномерным},
если у него есть конечный базис. Число элементов этого
базиса называется {\it размерностью} пространства $L$ и обозначается $\dim L$.

\z Доказать, что определение размерности пространства $L$
корректно, то есть если пространство конечномерно, то все его базисы
равномощны.

\z	\p Пусть некоторое множество линейно независимых
векторов не есть базис. Тогда к нему можно добавить еще один вектор
так, что векторы нового множества останутся линейно независимыми.
	\p Пусть некоторое множество векторов не есть базис,
но его линейная оболочка совпадает со всем пространством.
Тогда из него можно убрать один вектор так,
что линейная оболочка при этом не изменится.


\z Доказать, что в конечномерном пространстве
	\p любое линейно независимое множество векторов можно дополнить до базиса;
	\p из всякого множества векторов, линейная оболочка которого
совпадает со всем пространством, можно выделить базис.

\z Какие из пространств задачи $\bf1\diamond1$ являются конечномерными? Укажите для них размерности и приведите примеры базисов.

\z Пусть $L_1$ --- подпространство размерности $k$ конечномерного пространства $L$. Тогда в $L$ можно выбрать такой базис, что $k$
его векторов будут лежать в $L_1$.

\z Пусть $L_1$, $L_2$ --- подпространства конечномерного пространства $L$.
Доказать, что $\dim L_1 + \dim L_2 = \dim(L_1 + L_2 ) + \dim(L_1 \cap L_2)$.




\end{document}






