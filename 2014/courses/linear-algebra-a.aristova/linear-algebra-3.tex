\documentclass[12pt]{article}
\usepackage[utf8]{inputenc}
\usepackage[english,russian]{babel}
\usepackage{konduit01-utf8}

\textheight=210mm

\newcommand{\pstar}[1][]{\refstepcounter{pnum}{\immediate\write\tempfile{\arabic{znum}.\arabic{pnum}.\pstyle.\if\relax\detokenize{#1}\relax\zprev\else#1\fi}} {\bf\thepnum}$^{\mbox{#1}*}${\bf{)}$\;$}}

%\newcommand{\Hom}{\rm{Hom}}
\DeclareMathOperator{\Hom}{Hom}
\newcommand{\image}{{\rm{im}\,}}
\newcommand{\Iso}{{\rm Iso}}
\newcommand{\Aut}{{\rm Aut}}
\newcommand{\id}{{\rm{id}}}
%\DeclareMathOperator{\im}{im}

\begin{document}

\name[Линейные отображения, матрицы]{3}{}{}

\defin Пусть $L_1$, $L_2$ --- линейные пространства. Отображение линейных пространств $A\colon L_1 \rightarrow L_2$ называется линейным отображением (гомоморфизмом), если выполняются следующие
условия: $A(x + y) = A(x) + A(y)$, $A(\lambda x) = \lambda A(x)$.
Множество всех линейных отображений из $L_1$ в $L_2$ обозначается $\Hom(L_1 , L_2)$.

\z Являются ли линейными следующие отображения
$A\colon L_1 \rightarrow L_2$:
	\p $Ax = 0$
	\p $L_1 = L_2$, $Ax = x$ (такое отображение называется {\it тождественным}; обозначение: $\id$ или $E$);
	\p $L_1 = \R^4$, $L_2 = \R^3$, $A(x, y, z, t) = (x + y, y + z, z + t)$
	\p $L_1 = L_2 = \R^3$, $A(x, y, z) = (x + 1, y + 1, z + 1)$
	\p $L_1 = L_2 = \R[x]$, $(Ap)(x) = p(\lambda x^2 + \nu)$, $\lambda,\nu\in\R$
	\p $L_1 = L_2 = \R[x]$, $(Ap)(x) = q(x)\cdot p(x)$, $q\in\R[x]$
	\pstar $L_1$ --- пространство сходящихся последовательностей действительных чисел, $L_2=\R$, $A(x) = \lim\limits_{i\rightarrow\infty}x_i$

\z Доказать, что $\Hom(L_1, L_2)$ --- линейное пространство относительно следующих операций: $(A + B)x = Ax + Bx$, $(\lambda A)x = \lambda(Ax)$.

\z Доказать, что произведение (композиция) линейных
отображений есть линейное отображение.

\defin {\it Ядром} линейного отображения $A$ называется
множество, состоящее из всех таких $x$, что $Ax = 0$. Обозначение: $\ker A$.
Образ линейного отображения $A$ обозначается $\image A$.

\z Доказать, что ядро и образ линейного отображения являются линейными пространствами.

\z Найти ядра и образы линейных отображений задачи $1$.

\z Пусть $A$ --- отображение пространства многочленов степени не выше $n$
с действительными коэффициентами в пространство
функций на $M \subset \R$, которое переводит многочлен в его ограничение
на $M$.
	\p Доказать, что $A$ линейно.
	\p При каких $M$ $\ker A = 0$?




\defin Отображение $A \in \Hom(L_1, L_2)$ называется изоморфизмом,
если $\ker A = 0$ и $\image A = L_2$.
Множество изоморфизмов обозначается $\Iso(L_1, L_2)$.
В случае $L_1 = L_2$ изоморфизмы называются автоморфизмами.
Обозначение: $\Aut(L_1)$.

\z Пусть $A \in \Hom(L_1 , L_2)$. Доказать, что следующие утверждения эквивалентны:
	\p $A$ --- изоморфизм;
	\p A взаимно однозначно;
	\p A обратимо, т. е. существует такое отображение $A^{-1} \in \Hom(L_2, L_1)$, что $AA^{-1} = \id$ и $A^{-1} A = \id$.

\z Пусть $A \in \Iso(L_2 , L_3)$, $B \in \Iso(L_1, L_2)$, $\lambda$ --- число, не равное нулю.
Доказать, что $\lambda A \in \Iso(L_2 , L_3)$, $AB \in \Iso(L_1, L_3)$,
и выразить обратные к $\lambda A$ и $AB$ отображения через $A^{-1}$ и $B^{-1}$.




\defin Пусть даны базис $(e_1, \dots, e_m)$ линейного пространства $L$
и базис $(g_1, \dots, g_n)$ линейного пространства $M$.
Пусть $A$ --- линейное отображение из $L$ в $M$, $Ae_i = \sum a_i^j g_j$.
Тогда набор чисел $(a_i^j )$,
записываемый в виде таблицы с $m$ столбцами и $n$ строками, называют
{\it матрицей отображения} $A$ в базисах $(e_i)$, $(g_j)$.
Если $L = M$, $(e_i) = (g_j)$, то
говорят о матрице оператора в базисе $(e_i)$.
Наконец, просто матрицей
называют прямоугольную таблицу чисел (элементов поля).

\z Доказать, что отображение, сопоставляющее линейному
отображению его матрицу в фиксированных базисах, взаимно однозначно.

\z Найти матрицы отображений задачи $1$.

\z Пусть $L$ --- пространство многочленов степени не выше
$n$ с действительными коэффициентами. Доказать, что следующие
отображения являются линейными, и найти их матрицы в базисе
$(x_n, \dots, x, 1)$:
	\p $Ap(x) = p(cx)$;
	\p $Ap(x) = p(x + s)$.

\z Пусть $L_1$, $L_2$, $L_3$ --- конечномерные линейные пространства,
в которых заданы базисы.
Пусть $(a_i)$, $(b_i)$, $(c_n)$ --- матрицы отображений
$A, B \in \Hom(L_1 , L_2)$,
$C \in \Hom(L_2, L_3)$ в этих базисах.
Найти матрицы следующих отображений:
	\p $\id$ (единичная матрица, обозначение: $\delta_i^j$ или $E$);
	\p $\lambda A$;
	\p $A + B$;
	\p $CA$ (правило <<строка на столбец>>).



\z \p Записать оператор $Rot_\alpha$ поворота плоскости на угол $\alpha$ матрицей.
	\p Проверить, что $Rot_\alpha Rot_\beta = Rot_{\alpha+\beta}$

\z Найти матрицу оператора $A^n$, если матрица оператора $A$ имеет вид
	\p $\begin{pmatrix}2 & 1 \\ 0 & 2 \end{pmatrix}$
	\p $\begin{pmatrix}0&1&0 \\ 0&0&1 \\ 1&0&0 \end{pmatrix}$
	\p $\begin{pmatrix}\lambda_1&0&0 \\ 0&\lambda_2&0 \\ 0&0&\lambda_3 \end{pmatrix}$
	\p $\begin{pmatrix}\lambda&1&0 &\dots&0&0 \\ 0&\lambda&1 &\dots&0&0 \\ 0&0&\lambda &\dots&0&0 \\ \dots&\dots&\dots&\dots&\dots&\dots \\ 0&0&0&\dots&\lambda&1 \\ 0&0&0&\dots&0&\lambda \end{pmatrix}$
	\p $\begin{pmatrix}\cos\alpha & -\sin\alpha \\ \sin\alpha & \cos\alpha \end{pmatrix}$
	\p $\begin{pmatrix}\ch\alpha & \sh\alpha \\ \sh\alpha & \ch\alpha \end{pmatrix}$

\end{document}






