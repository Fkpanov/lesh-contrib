\documentclass[12pt]{article}
\usepackage[utf8]{inputenc}
\usepackage[english,russian]{babel}
\usepackage{konduit01-utf8}

\textheight=210mm

\newcommand{\pstar}[1][]{\refstepcounter{pnum}{\immediate\write\tempfile{\arabic{znum}.\arabic{pnum}.\pstyle.\if\relax\detokenize{#1}\relax\zprev\else#1\fi}} {\bf\thepnum}$^{\mbox{#1}*}${\bf{)}$\;$}}

\begin{document}

\name[Ещё немного логики]{2}{}{}

\z  Докажите, пользуясь строгим определением эквивалентности, что следующие пары утверждений эквивалентны.
	\p $\lnot(A\land B)$; $\lnot A \lor \lnot B$ %Не (А и В); не А или не В.
	\p $\lnot(A\lor B)$; $\lnot A \land \lnot B$ %Не (А или В); не А и не В.
	%%\p $A\implies B$
	
\z (Ещё немного отрицаний). Назовём контрольую простой, если за каждой партой хотя бы один ученик решил все задачи. Дайте определение сложной контрольной.

\z Рассмотрим два определения легкой контрольной: 1) в каждом варианте каждую задачу решил хотя бы один ученик; 2) в каждом варианте хотя бы один ученик решил все задачи. Может ли контрольная быть легкой в смысле определения 1) и трудной в смысле определения 2)?

\z На острове живут рыцари и лжецы. Лжецы всегда лгут, рыцари всегда говорят правду. Островитянин A говорит:
а) «Я лжец или B рыцарь».
б) «По крайней мере один из нас лжец»
в) «Если я рыцарь, то B - лжец»
Кто из двух персонажей A и B рыцарь и кто лжец? 

\z В конференции участвовало 100 человек --- химиков и алхимиков. Каждому был задан вопрос: «Если не считать Вас, то кого больше среди остальных участников --- химиков или алхимиков?». Когда опросили 51 участника, и все ответили, что алхимиков больше, опрос прервался. Алхимики всегда лгут, а химики говорят правду. Сколько химиков среди участников? 

\z      Две коробочки помечены: А и В. Надпись на коробочке А гласит: «Надпись на коробочке B верна и золото в коробочке А». Надпись на коробочке B гласит: «Надпись на коробочке А не верна и золото в коробочке А». Зная, что в одной из коробочек лежит золото, скажите, в какой именно. (Считая, что утверждение на каждой из коробочек может быть либо истинным, либо ложным).

\z \p Через какие пары операций из следующего набора (<<и>>; <<или>>; <<не>>; <<следует>>) можно получить все таблицы истинности $2\times2$ для утверждений А и В?
	\pstar Придумайте одну логическую операцию, с помощью которой можно получить все таблицы истинности $2\times2$.
\end{document}
