\documentclass[12pt]{article}
\usepackage[utf8]{inputenc}
\usepackage[english,russian]{babel}
\usepackage{konduit01-utf8}

\textheight=210mm

\begin{document}

\name[Логика и множества]{1}{}{}

\z Какие из следующих утверждений эквивалентны? Какие следуют из каких?
	\p В огороде бузина.
	\p В Киеве нет дядьки.
	\p В огороде нет бузины.
	\p Если в огороде бузина, то в Киеве дядька.
	\p Если в огороде нет бузины, то в Киеве дядька.
	\p Если в Киеве нет дядьки, то в огороде нет бузины.
	\p Если в Киеве дядька, то в огороде бузина.

\z Сформулируйте отрицание к следующим утверждениям:
	\p Если бы на Марсе были города, я бы встал пораньше и слетал туда. (На Марсе есть города $\Rightarrow $ я встаю пораньше и я лечу туда)
	\p {\it(Основной постулат женской логики)} $\forall$ утверждения A и $\forall$ утверждения B если A $\Rightarrow$ B и B приятно, то A - истинно.
	\p $\forall$ охотника $\exists$ фазан, такой, что $\forall$ цвета радуги, нравящегося охотнику, $\exists$ перо в хвосте фазана такого цвета.

\z Верно ли, что для любых множеств A, B и C
	\p A$\setminus$(A$\setminus$B)=A$\cap$B
	\p A$\setminus$(B$\cup$C)=(A$\setminus$B)$\cap$(A$\setminus$C)
	\p A$\setminus$(B$\cap$C)=(A$\setminus$B)$\cup$(A$\setminus$C)
	\p A$\setminus$(B$\setminus$C)=(A$\setminus$B)$\cup$(A$\cap$C)
	\p (A$\setminus$B)$\cup$(B$\setminus$A)=A$\cup$B?

\z Верно ли, что для любых утверждений A, B и C
	\p A и не (A и не B) $\Leftrightarrow$ A и B
	\p A и не (B или C) $\Leftrightarrow$ (A и не B) и (B и не C)
	\p A и не (B и C) $\Leftrightarrow$ (A и не B) или (A и не C)
	\p A и не (B и не C) $\Leftrightarrow$ (A и не B) или (A и C)
	\p (A и не B) или (B и не A) $\Leftrightarrow$ A или B?

\z Какие из следующих утверждений эквивалентны? 
	\p Если в огороде бузина, то в Киеве дядька и в Петербурге идёт дождь.
	\p Если в огороде нет бузины, то в Киеве нет дядьки или в Петербурге не идёт дождь.
	\p Или в Киеве дядька и в Петербурге идёт дождь, или в огороде нет бузины.
	\p Если в Киеве нет дядьки и в Петербурге не идёт дождь, то в огороде нет бузины.
	\p Если в огороде нет бузины, то в Киеве нет дядьки и в Петербурге не идёт дождь.
	\p Или в Киеве дядька, или в Петербурге не идёт дождь, или в огороде нет бузины.
	\p Если в Киеве нет дядьки или в Петербурге не идёт дождь, то в огороде нет бузины.
	\p Если в Киеве есть дядька и в Петербурге идёт дождь, то в огороде есть бузина.
	\p Или в огороде бузина, или в Киеве нет дядьки, или в Петербурге не идёт дождь.





\end{document}
