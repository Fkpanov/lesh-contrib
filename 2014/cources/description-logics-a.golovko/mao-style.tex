\usepackage[utf8]{inputenc}
\usepackage[russian]{babel}

\newif\ifft
% Раскомментируйте эту строчку, чтобы скомпилировать файл в режиме "для преподов" (с пометками на поляхи устными задачами):
%\fttrue

% Поля.
\oddsidemargin=-0.5in
\textwidth=\ifft6.8in\else7.3in\fi
\topmargin=-0.75in
\setlength{\textheight}{45\baselineskip}
\setlength{\textheight}{\baselinestretch\textheight}
\addtolength{\textheight}{\topskip}

% Колонтитул.
\makeatletter
\renewcommand{\@oddhead}
{\raisebox{0pt}[\headheight][0pt]{%
\vbox{\hbox to\textwidth{Дескрипционные логики \hfil \strut
\the\year
%2014
}\hrule}}%
}
\makeatother

% Записывайте задачи как "\z15) Докажите, что...".
\usepackage{amssymb}
\newcounter{zadacha}%[section]
\def\z#1){\par\vspace{6pt}\noindent\addtocounter{zadacha}{1}%
\textbf{$\lozenge\,$\arabic{zadacha}.} }

% Заметки на полях (отображаются только в режиме "для преподов"):
% \m{А эту задачу никто не решит!}
\newcounter{myMargin}
\newcommand{\m}[1]{\ifft $^{\star\mbox{-}\tiny\arabic{myMargin}}$\marginpar{\tiny $\star\mbox{-}\tiny\arabic{myMargin}$  #1\\ \hrule\addtocounter{myMargin}{1}}\else\fi}

% $a\hm+b\hm=c$, если возникла необходимость сделать перенос в формуле.
% На всякий случай.
\newcommand*{\hm}[1]{#1\nobreak\discretionary{}%
{\hbox{$\mathsurround=0pt #1$}}{}}

% definitions
\usepackage{amsthm}
\newtheorem{df}{Определение}

%
\usepackage{enumerate}

%\usepackage{titlesec}
%\titlespacing*{\chapter}{0pt}{-50pt}{20pt}

% Для ведомости на обратной стороне листа.
\newif\ifvedomost
\newcommand{\vedomost}[1]{\ifvedomost{#1}\fi}