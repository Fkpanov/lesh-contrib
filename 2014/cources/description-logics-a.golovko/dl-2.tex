\documentclass[12pt]{article}
\usepackage{amssymb}
\usepackage[utf8]{inputenc}
\usepackage[russian]{babel}

\newif\ifft
% Раскомментируйте эту строчку, чтобы скомпилировать файл в режиме "для преподов" (с пометками на поляхи устными задачами):
%\fttrue

% Поля.
\oddsidemargin=-0.5in
\textwidth=\ifft6.8in\else7.3in\fi
\topmargin=-0.75in
\setlength{\textheight}{45\baselineskip}
\setlength{\textheight}{\baselinestretch\textheight}
\addtolength{\textheight}{\topskip}

% Колонтитул.
\makeatletter
\renewcommand{\@oddhead}
{\raisebox{0pt}[\headheight][0pt]{%
\vbox{\hbox to\textwidth{Дескрипционные логики \hfil \strut
\the\year
%2014
}\hrule}}%
}
\makeatother

% Записывайте задачи как "\z15) Докажите, что...".
\usepackage{amssymb}
\newcounter{zadacha}%[section]
\def\z#1){\par\vspace{6pt}\noindent\addtocounter{zadacha}{1}%
\textbf{$\lozenge\,$\arabic{zadacha}.} }

% Заметки на полях (отображаются только в режиме "для преподов"):
% \m{А эту задачу никто не решит!}
\newcounter{myMargin}
\newcommand{\m}[1]{\ifft $^{\star\mbox{-}\tiny\arabic{myMargin}}$\marginpar{\tiny $\star\mbox{-}\tiny\arabic{myMargin}$  #1\\ \hrule\addtocounter{myMargin}{1}}\else\fi}

% $a\hm+b\hm=c$, если возникла необходимость сделать перенос в формуле.
% На всякий случай.
\newcommand*{\hm}[1]{#1\nobreak\discretionary{}%
{\hbox{$\mathsurround=0pt #1$}}{}}

% definitions
\usepackage{amsthm}
\newtheorem{df}{Определение}

%
\usepackage{enumerate}

%\usepackage{titlesec}
%\titlespacing*{\chapter}{0pt}{-50pt}{20pt}

% Для ведомости на обратной стороне листа.
\newif\ifvedomost
\newcommand{\vedomost}[1]{\ifvedomost{#1}\fi}
%\usepackage{first-indent}
\begin{document}

\subsubsection*{Логика $\mathcal{EL}$.}
В $\mathcal{EL}$ есть имена классов (<<концептов>>) $A_1, A_2, \dots$, класс $\mathrm T$ (<<thing>>),
имена ролей $r_1, r_2, \dots$, операция пересечения $\sqcap$, квантор существования $\exists$.

Классы строятся из $A_1,\dots$, $T$;
если $C$, $D$ --- классы, $r$ --- роль, то $C\sqcap D$ и $\exists r.C$ --- классы.

Определения в $\mathcal{EL}$ имеют вид $A\equiv B$ или $A\sqsubseteq B$,
где $A$ --- имя класса, $B$ --- класс.

$\mathcal{EL}$-терминология --- конечное множество определений,
в котором никакое имя не определяется дважды.
(Циклические определения возможны).

$\mathcal{EL}$-TBox --- это конечное множество вложений классов $C\sqsubseteq D$.


\subsubsection*{Семантика $\mathcal{EL}$.}
Интерпретация $\mathcal{I} = (\Delta^\mathcal{I}, \cdot^\mathcal{I})$:
$\Delta^\mathcal{I} \neq \emptyset$ --- домен,
$\cdot^\mathcal{I}$ сопоставляет каждому имени класса $A$
некоторое подмножество домена $A^\mathcal{I} \subseteq \Delta^\mathcal{I}$,
каждому имени роли $r$ --- бинарное отношение
$r^\mathcal{I} \subseteq \Delta^\mathcal{I} \times \Delta^\mathcal{I}$.

Интерпретация произвольного класса задаётся следующими правилами:
$(\exists r.C)^\mathcal{I} =
\{ x\in\Delta^\mathcal{I} | \exists y\in\Delta^\mathcal{I}
  \colon 
(x,y) \in r^\mathcal{I}, \, y\in C^\mathcal{I} \}$,
$(C\sqcap D)^\mathcal{I} = C^\mathcal{I} \cap D^\mathcal{I}$,
$T^\mathcal{I} = \Delta^\mathcal{I}$.

{\z) Выберем следующую интерпретацию:
$\Delta^\mathcal{I} = \{\mbox{школьники}\} \cup \{\mbox{курсы МаО-}2014\}$, \\
$A_0^\mathcal{I} = \{\mbox{курсы за 4 цикл}\}$, \\
$A_1^\mathcal{I} = \{\mbox{решающий эту задачу}\}$, \\
$r_0^\mathcal{I} = \{(x,y)\colon \mbox{школьник x ходил на курс y}\}$, \\
$r_1^\mathcal{I} = \{(x,y)\colon \mbox{школьник x сдал y}\}$, \\
$r_2^\mathcal{I} = \{(x,y)\colon \mbox{на курс x ходил школьник y}\}$, \\
$r_3^\mathcal{I} = \{(x,y)\colon \mbox{курс x был сдан школьником y}\}$. \\
Постройте интерпретации классов $\exists r_0.A_0$, $\exists r_1.A_1$, $\exists r_2.A_0$,
$\exists r_2.A_1$, $\exists r_3.A_0$, $\exists r_3.A_1$.
}

\subsubsection*{Модели и вывод в $\mathcal{EL}$.}
Будем говорить, что в интерпретации $\mathcal{I}$ выполняется вложение классов $C\sqsubseteq D$
(записывается $\mathcal{I} \vDash C\sqsubseteq D$), если
$C^\mathcal{I} \subseteq D^\mathcal{I}$.
%$C\equiv D \Leftrightarrow C\sqsubseteq D \,\mbox{и}\, D\sqsubseteq C$.
Аналогично, $\mathcal{I} \vDash C\equiv D \Leftrightarrow C^\mathcal{I} = D^\mathcal{I}$.

Пусть $\tau$ --- $\mathcal{EL}-TBox$. Интерпретация $\mathcal{I}$ называется моделью для $\tau$
($\mathcal{I} \vDash \tau$),
если в ней выполняется каждое вложение или равенство классов из $\tau$.

Говорят, что из TBox $\tau$ выводится вложение классов ($\tau \vDash C\sqsubseteq D$), если
$\forall\mathcal{I} \, \mathcal{I}\vDash\tau \Rightarrow \mathcal{I}\vDash C\sqsubseteq D$.

\z) Пусть $\tau = \{A \sqsubseteq B \sqcap C\}$.
Верно ли, что 
а) $\tau \vDash B \sqsubseteq C$?
б) $\tau \vDash B \sqsubseteq A$?
в) $\tau \vDash A \sqsubseteq B$?
г) $\tau \vDash \exists r . A \sqsubseteq \exists r . B$?
д) $\tau \vDash \exists r . B \sqsubseteq \exists r . A$?

%е) $\tau \vDash $
%ё) $\tau \vDash $
%ж) $\tau \vDash $
%з) $\tau \vDash $
%и) $\tau \vDash $

\z) Верно ли, что если $\tau\vDash C\sqsubseteq D$ и $\tau\vDash D\sqsubseteq E$, то
$\tau\vDash C\sqsubseteq E$?

\z) Пусть $\tau = \{\exists r.\mathrm{T} \sqsubseteq C,\, E \equiv \exists r.D\}$. % домен r = C
Верно ли, что $\tau \vDash E \sqsubseteq C$? $\tau \vDash \exists r.C \sqsubseteq C$?

%\z0) Куча упражнений на то, выводится ли одно из другого, что выводится, доказать, опровергнуть



\subsubsection*{Логика $\mathcal{ALC}$.}
% 4 слайд.
В логике $\mathcal{ALC}$ к уже известным классам, ролям, $\exists$, $\sqcup$, $T$
добавляются $\perp$ (пустой класс), $\sqcap$ (пересечение классов), $\forall$, $\lnot$
(отрицание, дополнение к классу).

\z5) Сформулируйте, как интерпретируются новые конструкции.



\pagebreak







\subsubsection*{Зоопарк дескрипционных логик.}
$\mathcal{AL} = T | \perp | A | \lnot A | C\sqcap D | \exists r.C | \forall r.C$.

В $\mathcal{ALC}$ добавляется $\sqcup$.

Семантика новых символов:
\begin{itemize}
\item $\perp^\mathcal{I} = \emptyset$
\item $(C\sqcup D)^\mathcal{I} = C^\mathcal{I} \cup D^\mathcal{I}$
\item $(\forall r.C)^\mathcal{I} = \{x\in\Delta^\mathcal{I}\colon \forall y\in\Delta^\mathcal{I}, (x,y)\in r \Rightarrow y\in C\}$
      (ограничение значения)
\item $(\lnot C)^\mathcal{I} = \Delta^\mathcal{I}\setminus C^\mathcal{I} = \{x\in\Delta^\mathcal{I}\colon x\notin C^\mathcal{I}\}$
\end{itemize}

$\mathcal{FL}^-$ --- это $\mathcal{AL}$ без отрицаний.

$\mathcal{FL}^0$ --- $\mathcal{AL}$ без отрицания и квантора существования.


$\mathcal{F}$ --- функциональность $T\sqsubseteq \leq 1 r.T$.

$\mathcal{N} = \leq n \, r.T,\,\geq n\,r.T$.

$\mathcal{Q} = \leq n \, r.C,\,\geq n\,r.C$ (ограничения кардинальности).

$\mathcal{S}$ --- ALC + транзитивность.

$\mathcal{I}$ --- обратные роли.

$\mathcal{H}$ --- включения ролей.

$\mathcal{R}$ --- композиция ролей $r\circ s$.

$\mathcal{O}$ --- номиналы (классы из ровно одного указанного элемента).


$\mathcal{SHOIQ}$ это OWL DL, $\mathcal{SROIQ}$ это OWL 2.

% f --- функция

\z) Запишите в виде вложений классов следующие утверждения:
а) каждый человек имеет ровно одного отца
б) слоны делятся на индийских и африканских
в) зачёт может быть сдан только по тому курсу, содержание которого человек изучил
г) если человек был на ЛЭШ-2014, то хотя бы в один из дней он был в Беляево
д) на любом костре либо есть взрослый, либо есть проблема
е) Лёша и Саша это два разных человека
ё) Карл Маркс и Фридрих Энгельс --- не муж и жена, а 4 совершенно разных человека
$\mbox{ж}^*$) вложение классов транзитивно.

\z) Реализуемы ли (то есть найдётся ли модель, в которой эти классы представлены непустыми множествами)
следующие классы?
а) $(\forall r.C) \sqcap (\forall r.\lnot C)$
б) $(\forall r.C) \sqcap (\exists r.\lnot C)$
в) $\lnot (\forall r.C) \sqcap (\exists r.\lnot C)$
г) $\lnot (\forall r.C) \sqcap \lnot (\exists r.\lnot C)$

\z) Пусть в TBox входят $T\sqsubseteq \leq 1\,r.T$, $T\sqsubseteq \geq 1\,r.T$, $r\circ r\sqsubseteq r$.
Что можно сказать о моделях этого TBox?

%\z) $C\equiv \exists r.E$, $D\equiv \exists r^-.E$, $r$ транзитивно, $r$ и $r^-$ имеют минимальную кардинальность 1.
%Верно ли, что $C\equiv D$?

%\z) Верно ли, что $$

\end{document}

% Соображения:
% про интерпретации важно рассказать, и там можно дать упражнения.
% вывод (через определение "в любой модели") важная штука, нужны упражнения и много
% Канва такая: определение EL, интерпретация, модель, вывод (2 слайд);
% зоопарк(5 слайд, начало 6); каноническая модель; табличный алгоритм(4 слайд, начало 5); бла-бла о тройках.
% 7 слайд -- о запросах. 8 — тоже, потом об OWL, DL-lite



% Без паники. Спокойно.
% Я планировал ещё зоопарк логик, табличный алгоритм, практикум и болтовня об RDF.
% По-моему, мы не успеем всё это.
% Поэтому я делаю зоопарк, даю упражнения на всякие кванторы и упражнения на формулировку аксиом.
% Потом будет практикум и болтовня.