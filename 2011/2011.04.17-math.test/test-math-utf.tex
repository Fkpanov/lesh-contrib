\documentclass[draft]{article}
\usepackage[simple]{dmvn}
\input epsf
\pagestyle{empty}

\begin{document}
\hbox to \textwidth{\Large Тест <<Семнадцать мгновений весны>>\hfil ФизЛЭШ, 17 апреля 2011~г.}
\medskip

%\hbox to \textwidth{\hfil\hbox to .6\textwidth{\vbox{\hsize.6\textwidth\small
{\small Мы никак не учитываем результаты теста при приеме на отделение, сей
тест нужен лишь для выяснения того, что мы можем считать известным при
построении своих курсов.

Мы обязуемся сохранить конкретные результаты каждого школьника
известными лишь тем, кому они нужны\т авторам курсов.\par}
%}\hfil}

\section{Задачи}

Помните, что решить задачу\т это не только угадать ответ,
но и~объяснить, откуда он взялся (обоснованием может быть
перебор вариантов, предварительные выкладки \итп).
Пожалуйста, не пользуйтесь вычислительной техникой (калькуляторами,
телефонами \итп) при~решении задач.

\def\po#1{\mbox{#1})\;}

\begin{problem}
Вычислите
$$
\po{a}\frac79+\frac45,\qquad
\po{b} 10^1+10^2+10^3,\qquad
\po{c} 25^{\frac32},\qquad
\po{d} \hr{16^c\cdot 16^d}^3.$$
\end{problem}

\begin{problem}
Имеет ли корни уравнение $x^2 + 2x + 10000=0$? Ответ обосновать.
\end{problem}

\begin{problem}
Решите уравнения
$$
\po{a} x=\sin 30^\circ,\qquad
\po{b} \sin 2x =1,\qquad
\po{c} \frac{\sin 2x}{\cos x}=2.$$
\end{problem}

\begin{problem}
В~каких точках отрезка $[0;2\pi]$ функция $\cos x + \sin x$ имеет максимум?
\end{problem}

\begin{problem}
Сравните числа и~обоснуйте ответ:
$$
\po a \sin \frac\pi{100}\quad\vee\quad 0.04\qquad
\po b \sqrt 3\quad\vee\quad \sqrt{5-\sqrt 2}.
$$
\end{problem}

\begin{problem}
Решите уравнение $13^{8x-1} = 1$.
\end{problem}

\begin{problem}
Нарисуйте схематически графики функций
$$
\po{a} f(x)=(x+2)^2,\qquad
\po{b} f(x)=x^3+1,\qquad
\po{c} f(x)=\hm{x}-3,\qquad
\po{d} f(x)=2^{x-1},\qquad
$$
\end{problem}

\begin{problem}
Как, не нарушая правил Московского метро, провезти удочку длиной $81$~см,
если запрещено провозить предметы, хотя бы по~одному из своих измерений (ширина, глубина, высота)
превышающие $50$~см?
\end{problem}

\begin{problem}
Сколько трёхзначных чисел, делящихся на~$3$ и~на~$5$, не делится на~$21$? Ответ обосновать.
\end{problem}

\begin{problem}
Решите уравнение $2^{x^2} = \cos x$.
\end{problem}

\begin{problem}
На~озере расцвела одна лилия. Каждый день количество лилий удваивалось,
и~на~10\д й день лилии покрыли цветками всё озеро. На~какой день лилиями
покрылась половина озера?
\end{problem}

\begin{problem}
В~семье двое детей. Известно, что один из них\т мальчик. Какова вероятность того, что
и~второй\т тоже мальчик? Вероятности рождения мальчика и~девочки считать равными.
\end{problem}

\begin{problem}
Тело массой $3.4$\,г брошено под углом $45^\circ$ к~горизонту со скоростью $900$\,$\frac{\mbox{м}}{\mbox{с}}$.
Найти дальность полёта. Ускорение свободного падения $g=9.8$\,$\frac{\mbox{м}}{\mbox{с}^2}$.
\end{problem}

\def\pz{\phantom{0}}

\begin{problem}
Найдите хотя бы три решения системы уравнений
$$\left\{\begin{aligned}
\pz1x+ \pz2y + \pz3z + \pz4w &=0,\\
\pz5x+ \pz6y + \pz7z + \pz8w &=0,\\
\pz9x+10y +11z + 12w &=0,\\
13x+14y +15z + 16w &=0.\\
\end{aligned}\right.$$
\end{problem}

\begin{problem}
Вычислите производные:
\eqn{\label{eq:ln}
\po a \frac{d}{dx}\sin x,\qquad
\po b \frac{\pd}{\pd x}(x^2+y^2+1),\qquad
\po c \frac{d}{dx} \ln \hr{\frac{x+5}{x-5}}.}
\end{problem}

\begin{problem}
Нарисуйте эскиз графика функции~$f(x)=\ln \hr{\frac{x+5}{x-5}}$.
\end{problem}

\begin{problem}
Вычислите (неопределённый) интеграл функции из предыдущей задачи.
\end{problem}

\newpage
\rightline{Ф.И.О. \vrule width 9cm height .4pt depth 0pt}

\section{Вопросы}

Пожалуйста, \emph{максимально честно} ответьте по~5\д балльной шкале на~приведённые
ниже несколько вопросов
про различные математические понятия. Баллы ставьте в~квадратике напротив понятия.
Не бойтесь ставить цифру~1 или~2! Помните,
что если Вы чего-то не знаете, то это \emph{не страшно}\т
Вас постараются этому научить. Нам важно знать Ваш уровень! Не забудьте подписать листок
сверху в~отведённом для этого месте.

Если Вы проходили и~усвоили какое-либо понятие, пожалуйста, \emph{напишите его определение}
ниже (можно своими словами, не пытайтесь механически
вспомнить заученное определение из~книжки). Если не хватит места\т не стесняйтесь писать на~обратной
стороне листа!

\medskip
\centerline{\textbf{Шкала оценок}}
\begin{nums}{-4}
\item Не проходили, не знаю
\item Проходили, но лучше объясните еще раз
\item Проходили, но не понял, зачем оно вообще нужно
\item Проходили в~школе, вроде понял
\item Знаю, умею использовать, могу рассказать другим
\end{nums}

\def\bx{\lower2mm\hbox{\epsfbox{pictures.20}}}
\def\ro#1{\leftline{\bx~#1}}

\centerline{\textbf{Понятия}}
\medskip

\hbox to\textwidth{\hbox to .5\textwidth{\vbox{\hsize.5\textwidth
\ro{функция}
\ro{корень уравнения}
\ro{квадратный корень}
\ro{корень $n$\д й степени}
\ro{квадратное уравнение}
\ro{степень числа}
\ro{степенная функция}
\vfil
}}%
\hbox to .5\textwidth{\vbox{\hsize.5\textwidth
\ro{показательная функция}
\ro{логарифм}
\ro{синус (косинус)}
\ro{производная}
\ro{частная производная}
\ro{вероятность}
\ro{интеграл}
\vfil
}}}

\newpage

\section{Запасные задачи}

\begin{problem}[<<Военный косинус>>, Юра Ц.]
Бывает ли такой угол, что его косинус равен $\frac{\pi}{4}$?
\end{problem}

\begin{problem}[Народная]
Нарисуйте эскиз графика функции
\eqn{\label{eq:sin}f(x)=x\sin \frac1x.}
\end{problem}

\begin{problem}[Юра Ц.]
Какова вероятность вынуть наугад два белых шара из корзины, в~которой лежат
$8$~белых и~$3$~чёрных шара?
\end{problem}

\begin{problem}[Народная]
В~прямоугольной плите вырезали квадратное отверстие (так, что вырез не задевает
края плиты, см.~рис.). Как с~помощью циркуля и~линейки разрезать плиту
на~две части одинаковой площади?

\centerline{\epsfbox{pictures.10}}
\centerline{\normalfont\footnotesize Рис.~1}
\end{problem}

\begin{problem}[<<Комары>>, Дима В.]
На~одном болоте Тверской области к~началу июня жило всего $100$~комаров
(по~$50$ особей каждого пола).
каждая самка откладывает по~$150$~яиц в~неделю, и~за~эту неделю из каждого
их них вылупляется новый комар.
при~этом $\frac1{10}$ всех вылупившихся комаров каждый день погибает по~естественным
и~другим причинам. Оцените (не обязательно очень точно) количество комаров на~болоте
к~началу Летней Школы ($24$~июля).
\end{problem}

\end{document} 