\documentclass[a4paper,12pt]{article}

% to make PDF searchable
\usepackage{cmap}

\usepackage{indentfirst}
\usepackage{fancyhdr}
\usepackage[T1,T2A]{fontenc}

% язык русский, кодировка utf8
\usepackage[utf8]{inputenc}
\usepackage[russian]{babel}

\usepackage{amsmath}
\usepackage[left=2cm,right=2cm,top=3cm,bottom=2cm]{geometry}
\usepackage{hyperref}
\usepackage{bookmark}

\headheight16pt

% Даты исправлять только здесь
\newcommand{\olympdatestart}{1~ноября}
\newcommand{\olympyearstart}{2015}

\newcommand{\olympdateend}{1~декабря}
\newcommand{\olympyearend}{2016}

\newcommand{\olympcheckend}{20~января}

\newcommand{\olympmail}{fizleshcontest@yandex.ru}
%\newcommand{\olympmail}{contest@fizlesh.ru}

%\newcommand{\olympspocmail}{\href{mailto:demarin@mail.ru}{demarin@mail.ru}}
\newcommand{\olympspocmail}{\href{mailto:lilienberg@rambler.ru}{lilienberg@rambler.ru}}
%\newcommand{\olympspocphone}{\textbf{+7(910)473-69-06} (Дмитрий)}
\newcommand{\olympspocphone}{\textbf{+7(985)111-21-15} (Иван)}


\pagestyle{fancy}
\fancyhead{}
\fancyhead[LO]{Межрегиональная физическая олимпиада \olympyearstart---\olympyearend. Правила}
\fancyhead[RO]{Стр.~\thepage~из 2}
\fancyfoot{}

\newcommand\hr[1]{\left({#1}\right)}
\newcommand\un[1]{\,\emph{#1}}

\def\thesection{\arabic{section}.}
\def\thesubsection{\arabic{section}.\arabic{subsection}.}

\newcommand{\ctxitems}[1]{%
\begin{enumerate}
\setlength{\itemsep}{-3pt}
#1
\end{enumerate}}

\title{\bf Межрегиональная физическая олимпиада\\Правила заочного тура}
\author{ГБОУ <<Интеллектуал>> г. Москвы --- Физическое Отделение ЛЭШ\\
fizlesh.ru/contest}

\begin{document}
\maketitle
\thispagestyle{empty}

Межрегиональная олимпиада состоит из заочного и очного туров. Первый тур проводится
% с~\textbf{\olympdatestart~\olympyearstart~года}
до \textbf{\olympdateend~\olympyearend~года} включительно.
До этого дня (не обязательно в последний день, лучше раньше) необходимо выслать решения предложенных задач.
Условия задач можно увидеть также на сайте \href{http://fizlesh.ru/contest}{fizlesh.ru/contest}
(они будут опубликованы там \textbf{\olympdatestart~\olympyearstart~года})
Там же можно найти более подробную информацию об олимпиаде и её организаторах.

\bigskip

Работу можно оформлять как в виде электронного документа, так и в письменном виде, на двойных тетрадных листах.
В любом случае в работе должны присутствовать:
\begin{enumerate}
\setlength{\itemsep}{-3pt}
\item Титульный лист (на нём не должно быть решений задач)
\item Решения задач теоретической части
\item Решения задач экспериментальной части
\end{enumerate}

Титульный лист заполняется в соответствии со следующими пунктами:
\begin{enumerate}
\setlength{\itemsep}{-3pt}
\item Фамилия, имя, отчество участника олимпиады (полностью, печатными буквами)
\item Фамилии, имена, отчества родителей (полностью)
\item Школа, класс
\item Домашний адрес полностью, с индексом, названием населённого пункта и региона
\item Контактный телефон
\item Действующий адрес электронной почты (крайне желателен для оповещения о приглашении на второй тур)
\item Название детского объединения (кружок, клуб) по физике, которое посещаете,\\
Ф.И.О. руководителей (полностью, печатными буквами)
\item Фамилия, имя, отчество учителя физики (полностью, печатными буквами)
\end{enumerate}

% new par
Задание состоит из теоретической и экспериментальной частей. В~первой части предлагается
дать максимально точный ответ, учтя наибольшее количество факторов, играющих роль в~каждой
задаче и~проявив знание необходимых законов. Во второй части нужно проявить умение практического
применения законов физики и~проверки моделей, а~также умение грамотно интерпретировать результаты
эксперимента. Важно, что в~этой части необходимо провести именно реальный, а~не~мысленный эксперимент.

\bigskip

Решать все задачи не обязательно. Лучше максимально полно ответить на вопросы задач,
рассмотреть интересные случаи. В некоторых задачах возможно несколько решений, базирующихся
на разных идеях.

\bigskip

Возможно использование литературы и~других источников информации (в~том числе сети Интернет).
Однако работа выполняется индивидуально, пользоваться помощью сверстников и~учителей не~разрешается.

\textbf{Обращаем Ваше внимание}, что в случае обнаружения признаков списывания друг у~друга,
иных форм <<коллективного творчества>> и~других нарушений Правил проведения Олимпиады,
Оргкомитет оставляет за~собой право дисквалифицировать участников, прошедших по сумме баллов
во~второй тур.

\bigskip

\input formatting

\bigskip

\olympcheckend~\olympyearend~года закончится проверка работ и определение победителей первого тура.
Победители будут приглашены на второй тур олимпиады~--- Всероссийскую Весеннюю Многопредметную Школу
в г.~Пущино. Списки приглашённых появятся на сайте
\href{http://fizlesh.ru/contest}{fizlesh.ru/contest}.
\textbf{Важно}: Будьте готовы достаточно оперативно ответить на письмо, которое может
прийти к Вам на e-mail в этих числах.

На Школе будет прочтено много интересных курсов и факультативов, проведены различные экскурсии.
Участники, победившие в олимпиаде по итогам второго тура, будут награждены призами и грамотами.

Самые заинтересованные и успешные участники будут приглашены
на Физическое отделение Летней Экологической Школы.

\end{document}
