\documentclass[12pt, onecolumn]{report}
\pagestyle{empty}

\usepackage{amssymb}
\usepackage[russian]{babel}
\usepackage{amsmath}
\usepackage{amsfonts}
\usepackage{latexsym}
\usepackage{amsthm}
\usepackage{graphicx}
\usepackage[utf8]{inputenc}

\newtheorem*{lemma}{Лемма}
\newtheorem{definition}{Определение}
\newtheorem*{remark}{Замечание}
\newtheorem*{theorem}{Теорема}
\newtheorem*{exe}{Упражнение}
\newtheorem*{lproof}{Доказательство}

\begin{document}
\chapter*{Для начала}
\section{Как зовут преподавателя}
Я бы на вашем месте сразу забыла, как меня зовут, поэтому напишу шпаргалку. Меня зовут Лена. Или Ася, так получилось, на Асю я даже охотнее откликаюсь. Но поскольку среди преподавателей МаО уже есть и Лена, и Ася, меня обозвали Леной-Асей.

Если кто-то пожелает, скажем, зафрендить меня вконтактике, то я вполе нахожусь по следующим данным: Лена Кожина, Москва, НИУ ВШЭ (ГУ-ВШЭ), факультет математики. Я открыта для диалога: если не буду отсыпаться после экзамена (или просто так) и не окажусь где-нибудь в Украине без связи и денег, то с удовольствием поболтаю с вами о математике в целом, теории множеств в частности (то и другое --- в меру своих скромных знаний), их высшем предназначении и вообще. А может, и на более отвлечённые темы. 
\section{Что можно почитать}
Н. Я. Виленкин, <<Рассказы о множествах>>. Оно есть здесь: \\ http://ilib.mccme.ru/pdf/rasomn.pdf
\chapter{А теперь перейдём к делу, или День первый}
\section{Определение множества}
А его нет. Простите, так уж вышло.

{\itshape Множество есть совокупность различных элементов, мыслимая как единое целое}, говорит нам Бертран Рассел  (точнее, говорит википедия, что так говорил Бертран Рассел). Как вы можете заметить, это какая-то философия. А всё потому, что множество --- это чересчур уж фундаментальное понятие, настолько, что воспринимается аксиоматически. То есть на веру. И вам придётся просто поверить.

\section{Но что всё-таки это такое}
На самом деле, конечно, всем интуитивно понятно, что множество дней недели --- это понедельник, вторник и так далее до воскресенья, а множество натуральных чисел --- это 1, 2 и так далее до бесконечности, и если уж множество $A$ состоит из $x$, $y$ и $z$, то из них оно и состоит. В таком случае $x$, $y$ и $z$ будут называться элементами множества $A$, а понедельник и прочие вторники --- элементами множества дней недели. Множество конечное, если в нём конечное число элементов, и бесконечное, если бесконечное.

Да, форма записи:

множество дней недели = \{понедельник, вторник, \ldots , пятница\}

$A = \{x, y, z \}$

множество натуральных чисел = $\{0, 1, 2, 3, \ldots \} = \mathbb N$ 

$ \mathbb N$ --- это стандартное обозначение для множества натуральных \linebreakчисел. Да-да, неожиданный ноль. Простите, в теории множеств удобно считать ноль натуральным числом. В этом нет ничего страшного, во Франции вот все так живут.

\section{Если честно}
Так уж вышло, что теории множеств нет дела до понедельников. И любые свойства множества, связанные с тем, что оно состоит из понедельника и вторника, а не икса и игрека, нас в рамках этого курса по большому счёту волновать не должны. С другой стороны, называть множество дней недели и множество из семи различных латинских букв совпадающими как-то язык не поворачивается. Поэтому их называют изоморфными.

О, ну наконец-то умное слово.
\section{Изоморфизм, или давайте уже что-нибудь определим}
\begin{definition} Два множества называются изоморфными, если элементы одного из них находятся во взаимно однозначном соответствии с элементами другого. Иначе говоря, два множества $A$ и $B$ изоморфны, если каждому элементу из $A$ мы можем сопоставить элемент из $B$, причём разным элементам $A$ сопоставляются разные элементы $B$, и каждый элемент $B$ сопоставлен какому-то элементу из $A$.

Изоморфизм будем обозначать так: $A \sim B$.
\end{definition}
Это было такое двойное определение (потому что я ленивая): одновременно изоморфизма и взаимно однозначного соответствия. Это самое соответствие ещё называется страшным словом биекция, которое хорошо и приятно запомнить, но я всё-таки постараюсь не употреблять его слишком часто.

Давайте лучше вести беседы об изоморфизмах. Например, несложно заметить, что множество $A$ = множество из четырёх яблок = \linebreak  $\{{apple}_1, {apple}_2, {apple}_3, {apple}_4\}$ изоморфно множеству $B$ = $\{1, 2, 3, 4\}$, потому что мы можем пронумеровать четыре яблока четырьмя натуральными числами (это и есть взаимно однозначное соответствие). Следующей ступенью осознания должно стать понимание того, что любое множество из четырёх элементов изоморфно множеству из четырёх яблок, равно как и множеству из четырёх различных натуральных чисел. Более того, <<четыре>> здесь можно смело заменять на любое натуральное число $n$. Грубо говоря, всё, что нас интересует во множестве --- это количество элементов.

\begin{remark} Кстати, здесь можно вспомнить, что мы договорились считать ноль натуральным числом, поэтому это самое n вполне может быть равно нулю. Да, множество из нуля элементов нас тоже интересует. Его называют пустым множеством и обозначают так: $\varnothing$. Пустое множество является подмножеством любого другого.
\end{remark}
 Хотя стоп, мы же ещё не давали определение подмножества, официально вы ещё не знаете, что это такое.
\section{Подмножество}
\begin{definition}
Множество A является подмножеством множества B, если каждый элемент, принадлежащий A, принадлежит также и B. Обозначается это так: $A \subset B$ (или $B \supset A$).
\end{definition}
Есть ещё одно важное обозначение, которое мы почему-то до сих пор не ввели: $a \in B$ означает, что a является элементом B. Смотрите, я зачем-то написала a малое и B большое. Это, увы, не несёт в себе никакого скрытого смысла. Очень хотелось бы порадовать вас новостью о том, что прописными буквами принято обозначать множества, а строчными --- их элементы, но, увы и ах, это неправда (хотя мы и постараемся по возможности делать именно так). Всё дело в том, что одни множества могут быть элементами других, и тогда непонятно, какую букву выбрать, чтобы их обозначить --- побольше или поменьше.

Да, раз уж мы заговорили о том, что одни множества могут быть элементами других, стоит на всякий случай уточнить разницу между $A \subset B$ и $A \in B$. В первом случае A является элементом B, а вот элементы A элементами B не являются; во втором случае --- наоборот. Поскольку эта фраза опять напоминает цитату из какого-нибудь греческого философа, можно попробовать вспомнить, что, когда мы натолкнулись на философию впервые, речь шла о том, как что-то воспринимается как единое целое. Если $A \in B$, A воспринимается как единое целое (и это целое --- элемент B); если же $A \subset B$, A воспринимается как множество элементов (и все эти элементы принадлежат не только A, но и B).

\section{Пересечения, объединения и разности}
\begin{definition}
X --- пересечение множеств, если X состоит в точности из всех элементов, лежащих одновременно во всех этих множествах. Y --- их объединение, если оно состоит в точности из всех элементов, лежащих хоть в каком-то из множеств. Z --- разность множеств A и B, если оно состоит в точности из всех элементов, принадлежащих A и не принадлежащих B.
\end{definition}

Опять интуитивно очевидное определение, так что здесь нас интересуют только обозначения. Так, пересечение множеств $A$ и $B$ обозначается $A \cap B$, их объединение --- $A \cup B$, разность --- $A\setminus B$.

Правда, иногда нас интересуют пересечения и объединения бесконечных множеств, и тут, конечно, выписать все множества и расставить между ними знаки не получится. Поэтому если X --- семейство множеств, то их перечечение и объединение обозначают так: $\bigcap X$ и $\bigcup X$, а, к примеру, пересечение множеств $A_0, A_1, A_2, \ldots$ обозначают так:
$\bigcap\limits_{i = 0}^{\infty}{A_i}$, их объединение, соответственно, --- $\bigcup\limits_{i = 0}^{\infty}{A_i}$. Если вы уже видели обозначение для суммы с буквой $\sum$, то непременно найдёте много схожего, потому что принцип, по которому строятся эти обозначения, точно такой же. 

Правда, с моей стороны было бы мило предварить рассказ о пересечениях бесконечного количества множеств занимательной историей о том, как вообще работать с бесконечностями, какие они бывают, как одна бесконечность может быть больше другой и всё такое. Об этом мы и поболтаем в следующий раз, а пока озадачу вас конечными пересечениями и объединениями.

\chapter*{Задачи на всякое конечное}
Очень простые и довольно скучные, признаюсь, задачи, во время решения и сдачи которых можно развлечься разве что рисованием кружочков, но которые, тем не менее, могут пригодиться в дальнейшем и для общего понимания.

Как рисовать кружочки, они же круги Эйлера
%круги Эйлера ли?
, все наверняка понимают, но всё-таки вот так:
%тем не менее, порисовать круги Эйлера

\includegraphics[scale=0.8]{euler.jpg}

%и повставлять задачи из матфаковских листочков и про детей, играющих в футбол и шахматы

\chapter{Счётность. День второй}
\begin{flushright}
Нужно что-то сделать с потолками.

{\itshape Никому не известный персонаж, изредка гененирующий разные совершенно бесполезные фразы, которые, как и многие другие бесполезные вещи, имеют обыкновение заседать у меня в голове.}
\end{flushright}
\section{Вот, например, множество}
Я, помнится, обещала что-нибудь сделать с бесконечностями, так что будем что-нибудь делать с бесконечностями.

Вот, например, множество натуральных чисел $\mathbb N$. Оно замечательно не только тем, что естественно, а ещё симпатичное и всем нам нравится, но и тем, что мы бессовестно используем его для определения количества элементов конечных множеств. Действительно, если мы можем взаимно однозначно сопоставить элементам какого-то множества первые n элементов множества $\mathbb N$, мы говорим про это самое какое-то множество, что в нём n элементов. В каком-нибудь идеальном мире, наверное, можно придумать последний, бесконечно большой элемент $\mathbb N$, называть его, например, <<натуральная бесконечность>> и обозначать как-нибудь так: ${\infty}_{\mathbb N}$.

В математике всё обычно и делается примерно так, как в идеальном мире: если для чего-то нужно придумать определение, то оно придумывается по аналогии с уже существующими. Поэтому всё, что я предположила про натуральную бесконечность, почти правда, за исключением того, что называют её счётной бесконечностью и обозначают обычно просто $\mathbb N$.

Да-да, само множество натуральных чисел обозначают так же, как и количество элементов в нём.

\section{Мощность}
А теперь мне надоело длинное словосочетание <<количество элементов>>, и я его чем-нибудь заменю.
\begin{definition}
Мощностью множества называется количество элементов в нём.
\end{definition}
Множества равномощны, когда в них поровну элементов, то есть существует взаимно однозначное отображение из одного в другое, то есть они изоморфны. Множества имеют разную мощность, если такого отображения не существует. Множество A мощнее множества B, если между ними не существует взаимно однозначного отображения, зато существует такое отображение, при котором каждому элементу B сопоставляется какой-то элемент из A (и в A, конечно, остаются ещё какие-то элементы).

Мощность множества A обозначается так: |A|.

Мы уже выяснили, что мощность множества из n элементов равна n, и $|\mathbb N| = \mathbb N$.

\section{Счётные множества}
\begin{definition}
Множества, равномощные $\mathbb N$, называются счётными.
\end{definition}
\begin{remark}
Иногда счётными называют не только множества, в которых $\mathbb N$ элементов, но и конечные множества тоже. Мы так делать не станем, потому что не хочется, а если нам потребуется обозвать одним термином все счётные и конечные множества одновременно, будем говорить о не более чем счётных множествах.
\end{remark}
Возьмём теперь множество всех чётных чисел (обозначим его $2\mathbb N$). Какое множество мощнее --- множество всех натуральных чисел или множество всех чётных чисел?

Правильный ответ, конечно, таков: никакое множество не мощнее, они очень даже изоморфны, и построить изоморфизм легко: нужно числу $a \in \mathbb N$ сопоставлять число $2a \in 2\mathbb N$. Аналогично, кстати, можно построить изоморфизм и с $3 \mathbb N$, и с $5\mathbb N$, и с $n\mathbb N$ при $n \neq 0$.

\begin{remark}
На этой ноте можно, конечно, возмутиться и сказать, что считать множества $\mathbb N$ и $n\mathbb N$ одним и тем же неприятно и не хочется (иначе говоря, что эквивалентность их кажется в некотором роде неестественной). В утешение можно разве что отметить, что существуют инструменты для сравнения равномощных множеств, которыми можно воспользоваться и убедиться, что в некотором смысле $n\mathbb N$ меньше $\mathbb N$ при $n \neq 1$. Однако этот некоторый смысл и эти инструменты, сколь бы ни были они занимательны, не очень интересуют нас в рамках данного курса, такие дела.
\end{remark}

Можно ещё взять множество $A =\mathbb N \cup \{a\}$ и понять, что оно тоже изоморфно $\mathbb N$, то есть $\mathbb N + 1 = \mathbb N$ (в данном случае и $\mathbb N$, и 1 --- обозначения мощностей, а не множеств; не обращайте внимания на это замечание, оно никому не нужно и преследует только формалистские цели). Действительно, мы можем строить изоморфизм, отображая $a$ в 0, а $b \in A$, отличное от $a$ --- в $b + 1$. Так же <<со сдвигом>> строится и взаимно однозначное отображение в $\mathbb N$ из $\mathbb N \cup \{a_0, a_1, a_2, \ldots , a_n\}$.

\begin{remark}
Мы обозначаем элементы множества, с которым объединяем $\mathbb N$, буквами, а не натуральными числами, чтобы пересечение этого самого множества с $\mathbb N$ было нулевым. То, что $\mathbb N \cup \{0, 1, \ldots n\} \sim \mathbb N$ очевидно просто потому, что эти множества в точности совпадают, поскольку и ноль, и так далее, и n --- элементы $\mathbb N$, и их объеднинение с $\mathbb N$ не даст нам ничего нового.
\end{remark}

Итак, мы обнаружили, что если $\mathbb N$ объединять с конечным множеством, получится счётное. Здесь должно быть понятно, что вместо $\mathbb N$ можно брать вообще любое счётное множество: мы считаем их эквивалентными.

Теперь давайте посмотрим, что будет, если объединять $\mathbb N$ со счётным множеством. Для верности можем даже выписать первые элементы:

\begin{flushleft}
\begin{tabular}{cccccccc}
\ 0 & 1 & 2 & 3 & 4 & 5 & 6 & $\ldots$ \\
\ $a_0$ & $a_1$ & $a_2$ & $a_3$ & $a_4$ & $a_5$ & $a_6$ & $\ldots$ \\
\end{tabular}
\end{flushleft}

А теперь выпишем их по-другому:
\begin{flushleft}
\begin{tabular}{ccccccccccccccc}
\ 0 & $a_0$ & 1 & $a_1$ & 2 & $a_2$ & 3 & $a_3$ & 4 & $a_4$ & 5 & $a_5$ & 6 & $a_6$ & \ldots \\
\end{tabular}
\end{flushleft}

Это мы просто прошлись по такому зигзагу.
%нарисовать зигзаг
А ещё можно записать вот так:

\begin{flushleft}
\begin{tabular}{ccccccccccccccc}
\ 0 & $a_0$ & 1 & $a_1$ & 2 & $a_2$ & 3 & $a_3$ & 4 & $a_4$ & 5 & $a_5$ & 6 & $a_6$ & \ldots \\
\ 0 & 1 & 2 & 3 & 4 & 5 & 6 & 7 & 8 & 9 & 10 & 11 & 12 & 13 & \ldots \\
\end{tabular}
\end{flushleft}

Можно заметить, что мы уже построили взаимно однозначное соответствие с $\mathbb N$: если мы будем сопоставлять числу из верхней строчки число из нижней, то у нас всё получится. Можно, конечно, и явно выписать изоморфизм из $\mathbb N$ в $A =\mathbb N \cup \{a_0, a_1, a_2, \ldots \}$:

$2x \in \mathbb N \rightarrow x \in A$

$2x+1 \in \mathbb N \rightarrow a_x \in A$.

(Но зачем?)

Есть, кстати, особенно приятный случай, в котором вместо $a_i$ мы подставляем целое отрицательное число. Получаем что-то такое:

\begin{flushleft}
\begin{tabular}{cccccccc}
\ 0 & 1 & 2 & 3 & 4 & 5 & 6 & $\ldots$ \\
\ $-1$ & $-2$ & $-3$ & $-4$ & $-5$ & $-6$ & $-7$ & $\ldots$ \\
\end{tabular}
\end{flushleft}
После этого можно радостно построить изоморфизм с $\mathbb N$ так, как мы это вообще-то уже сделали. Значит, $\mathbb Z \sim \mathbb N$, что, разумеется, крайне приятно. Да, $\mathbb Z$ --- стандартное обозначение для множества целых чисел.

Это замечательное открытие --- подходящий момент, чтобы решить несколько задач.

\chapter*{Задачи}
%вставить задачи про бесконечные гостиницы, клетчатый листок и всё такое
\chapter{Прямое произведение множеств, третий день}
\section{Маленькая формальность}
\begin{definition}
Прямым произведением множеств A и B называется множество упорядоченных пар $(a, b)$, где $a \in A, b \in B$. Обозначается прямое произведение так: $A \times B$.
\end{definition}
Упорядоченная пара, если что --- это когда $(a, b)$ и $(b, a)$ --- не одно и то же, если $a \neq b$. Таким образом, $A \times B$ и $B \times A$ вообще-то разные вещи. К нашему большому счастью, эти два прямых произведения изоморфны. Изоморфизм построить легко: достаточно поменять местами числа в скобочках, и всё как по волшебству встанет на свои места.
\begin{exe}
Проверьте, что от расстановки скобок тоже ничего особенного не зависит, когда всё интересует нас лишь с точностью до изоморфизма.
\end{exe}
\section{А теперь можно вернуться к рисованию}
Если вы умницы и порешали задачи про клетчатый листок, вы запросто докажете, что прямое произведение двух не более чем счётных множеств счётно. Ну и я, конечно, тоже докажу, чтобы вы не решили вдруг, что я на самом деле не умею, а только говорю, что умею.
%вставить доказательство
Здесь тоже бывает приятный случай. Например, возьмём и сопоставим паре (a, b) их частное $\frac{a}{b}$ (разумеется, b не должно быть равным нулю). После чего нарисуем такую табличку:

\begin{flushleft}
\begin{tabular}{c|cccccc}
\ & 0 & 1 &$ -1$ & 2 & $-2$ & \ldots \\
\hline
\ 1 & (0, 1) & (1, 1)& $(-1, 1)$ & (2, 1) & $(-2, 1)$ & $\cdots$ \\
\ 2 & (0, 2) & (1, 2) & $(-1, 2)$ & (2, 2) & $(-2, 2)$ & $\cdots$ \\
\ 3 & (0, 3) & (1, 3) & $(-1, 3)$ & (2, 3) & $(-2, 3)$ & $\cdots$ \\
\ \vdots & \vdots & \vdots & \vdots & \vdots & \vdots & $\ddots$ \\
\end{tabular}
\end{flushleft}

Множество таких пар счётно, как мы уже выяснили. А ещё можно вспомнить, чему мы там сопоставляли пару (a, b) и понять, что среди этих пар встречаются все рациональные числа (множество рациональных чисел традиционно обозначается $\mathbb Q$). Правда, некоторые (а если точнее, то все) рациональные числа встречаются в нашем множестве пар больше одного раза (а если точнее, то бесконечно много раз).

С другой стороны, легко понять, что бесконечное подмножество счётного множества счётно (просто возьмём в этом самом бесконечном подмножестве первый элемент, затем второй, третий\ldots ), а значит, множество $\mathbb Q$ тоже получается счётным. И это, друзья мои, ещё одна маленькая, но приятная победа. Кхм, так о чём я.

Да, маленькое отступление, которое может показаться исключительно лирическим, но на самом деле имеет к жизни прямое отношение. Вот мы тут поговорили о прямом произведении двух счётных множеств. Как известно, если с двумя штуками можно сделать что-то такое, после чего получится одна такая же (например, прямо перемножить), то это самое можно сделать с любым конечным количеством (по индукции: сделаем с двумя --- станет на единицу меньше). А вот про бесконечности такое, конечно, говорить уже нельзя. Поэтому сразу после того, как мы прямо перемножили два счётных множества, нужно побеспокоиться, а как же перемножать счётное количество счётных множеств. Впрочем, это уже другая история. Давайте пока что-нибудь порешаем для успокоения нервов.
\chapter*{Задачи про прямое произведение}
%рассказать о рисовании
%вставить задачи про бесконечные гостиницы и на проверку всяких свойств

\chapter{Континуум. День четвёртый}
\section{Но как, но почему}
Серьёзно, давайте ещё подождём пару глав, я вам сейчас лучше каких-нибудь  звёздочек подкину. Хотя бы потому, что мне надоело с вами беседовать и ни словом не обмолвиться о континууме. А ведь такое красивое название.

Так вот, континуум (от continuum --- непрерывный, ну вы сами поняли) --- это, например, множество точек на прямой. Или на интервале (0, 1). Или на любом другом интервале:
%вставить картинку

И, конечно, все прочие множества, равномощные множеству всех точек на прямой, тоже замечательным образом континуальны. Множество всех точек на прямой имеет адекватное название, и вы его знаете: это множество вещественных чисел (они же действительное). Обозначается оно $\mathbb R$.

\section{Континуум не счётен*}
Пункт со звёздочкой, потому что это доказательство, пожалуй, чуть менее тривиально, чем остальные в этом курсе (хотя они все несложные, и сравнивать трудно).

Континуум, воистину, более чем счётен, что мы сейчас и докажем --- разумеется, от противного, разумеется, предположив, что множество вещественных чисел от 0 до 1 (для верности выкинем отсюда и 0, и 1. Все, конечно, понимают, что отрезок и полуинтервал --- такие же континуумы, как и интервал, но формально мы этого ещё не сказали) можно выписать в столбик, попробовав и обломавшись.

Чтобы обломиться, будем рассматривать всевозможные бесконечные вправо последовательности из нулей и единиц, в которых нет бесконечного хвоста из одних единиц.

История, в общем, такая, про которую на всяких олимпиадах можно писать <<школьный факт>> и не доказывать. Дело в том, что все рациональные числа --- это в точности все периодические $l$-ичные дроби, если, конечно, этот период не выглядит как $((l-1))$ (дело в том, что в десятичной системе счисления $0,9999\ldots$ неотличимо от $1,0000\ldots$: действительно, если вы возьмёте любое другое число, оно будет либо больше обоих этих чисел, либо меньше их обоих; с $l$-ичной системой счисления такая штука тоже работает, конечно).

Доказывается это легко: во-первых, если число рационально, то есть представимо в виде $\frac{a}{b}, a \in \mathbb Z, b \in \mathbb N, b \neq 0$, то мы можем с большим удовольствием поделить a на b столбиком. После того, как мы что-то на b поделили, у нас получается какой-то остаток по модулю b, и к нему мы дописываем цифру. Причём с определённого момента (когда в a заканчиваются цифры) мы дописываем только нули. Но поскольку остатков по модулю b конечное количество, однажды мы напишем то, что уже писали, а поскольку дописываем мы нули, с этого момента всё пойдёт точно так же, как уже было, и зациклится.

С другой стороны, если у нас есть число $\overline{a_1 a_2 a_3 \ldots a_n (b_1 b_2 b_3 \ldots b_m)}$ (черта означает, что это цифры, а не умножение), то это же просто сумма $\overline{a_1 a_2 a_3 \ldots a_n}$ и какой-то геометрической прогрессии с маленьким положительным делителем, а считать сумму геометрической прогрессии с $0 < q < 1$, да так, чтобы получалось частное целых ненулевых чисел, мы отлично умеем.

Это был школьный факт про рациональные числа; про действительные тоже бывает факт, и тоже, наверное, школьный. Все действительные числа --- это в точности все $l$-ичные дроби, кроме, конечно, тех, которые с хвостами из $(l-1)$. Они, эти действительные числа, примерно так и определяются.

Ну вот, поговорили о школьной программе, полдела сделано. Осталось понять, что все эти односторонне бесконечные последовательности, которые мы хотим рассматривать --- это такая дробная часть вещественного числа в его двоичной записи, всё, что есть после запятой (потому что ставить ещё и запятые мы поленились); ну разве что мы ещё не рассматриваем последовательность из одних нулей. 

Ура, мы как-то красиво записали числа на интервале $(0,1)$. Теперь мы хотим, видимо, попытаться записать их в столбик, бесконечный вниз, и пронумеровать натуральными числами --- как и планировали. Ну что ж, давайте пытаться.

\begin{flushleft}
\begin{tabular}{ccccccc}
\ 0. & $x_{0,0}$ & $x_{0,1}$ & $x_{0,2}$ & $x_{0,3}$ & $x_{0,4}$ & $\cdots$ \\
\ 1. & $x_{1,0}$ & $x_{1,1}$ & $x_{1,2}$ & $x_{1,3}$ & $x_{1,4}$ & $\cdots$ \\
\ 2. & $x_{2,0}$ & $x_{2,1}$ & $x_{2,2}$ & $x_{2,3}$ & $x_{2,4}$ & $\cdots$ \\
\ 3. & $x_{3,0}$ & $x_{3,1}$ & $x_{3,2}$ & $x_{3,3}$ & $x_{3,4}$ & $\cdots$ \\
\ 4. & $x_{4,0}$ & $x_{4,1}$ & $x_{4,2}$ & $x_{4,3}$ & $x_{4,4}$ & $\cdots$ \\
\ \vdots & \vdots & \vdots & \vdots & \vdots & \vdots & $\ddots$ \\
\end{tabular}
\end{flushleft}
Здесь, как несложно догадаться, первый индекс --- номер последовательности в столбце, второй --- номер цифры в последовательности. Напоминаю, что каждый икс, как бы страшно он ни выглядел, на самом деле всего лишь ноль или единица.

Отлично. Теперь возьмём такую приятную последовательность: 

$$\overline{x_{0,0} x_{1,1} x_{2,2} x_{3,3} x_{4,4} \ldots}$$

Или лучше даже такую последовательность:

$$\overline{(1-x_{0,0})(1-x_{1,1})(1-x_{2,2})(1-x_{3,3})(1-x_{4,4}) \ldots}$$
От предыдущей она отличается во всех знаках (там, где в предыдущей 0, здесь 1, и наоборот). В этот самый момент я имею заявить, что вторая последовательность ни разу не была выписана в нашем столбике. Действительно, пусть она была выписана под каким-то номером n. Но тогда они должны совпадать во всех цифрах, в том числе и в энной, то есть $x_{n,n} = (1-x_{n,n})$, что, конечно, неправда, когда x --- это ноль или единица. Противоречие. Выходит, выписать все интересующие нас последовательности мы не могли. Это и означает, что континуум более чем счётен. А это, в свою очередь, означает победу, ура, конец доказательства.

\section{Континуум-гипотеза}
Континуум-гипотезу сформулировал Георг Кантор (всё остальное сделал тоже он, но мы почему-то вспомнинаем его имя только теперь. Прости, Кантор) и всё пытался доказать. И не смог. И неудивительно.

Дело в том, что, коль скоро множество задаётся аксиоматически, в теории множеств существует своя система аксиом, которые я от вас утаила, потому что они скучные. Она называется системой аксиом Цермело--Френкеля, и, как доказал Гёдель в 1940 году (это, во-первых, через двадцать два года после смерти Кантора, во-вторых, совсем недавно), в этой самой системе аксиом {\itshape отрицание} континуум-гипотезы доказана быть не может. Жалких пятьдесят лет назад, в 1963, Коэн доказал и то, что сама континуум-гипотеза в системе аксиом Цермело--Френкеля недоказуема.

Так она и повисла где-то вне этого бренного мира. В связи с чем хочу предупредить вас от одной распространённой ошибки.

Друзья мои, нельзя, нельзя пользоваться континуум-гипотезой! Если даже вы доказали, что ваше множество более чем счётно, но не более чем континуально, это отнюдь не значит, что оно континуум. Это, как обнаружили Гёдель, Коэн и теперь вы, вообще значит непонятно что.

\chapter*{И немного задач}
%задачи про изоморфизм между отрезком и интервалом, квадратом и кругом и всё такое

\chapter{Пятый день и теорема Кантора--Бернштейна}
\section{Что это такое}
Теорема Кантора--Бернштейна нужна нам только для того, чтобы поддержать благое начинание и упомянуть побольше фамилий.

Это, если что, неправда. На самом деле теорема Кантора--Бернштейна, как бы очевидно она не звучала, очень облегчает решение многих задач, например, вчерашних про континуумы и всякие геометрические фигуры. Именно поэтому я привожу её ближе к концу, когда вы уже над ними помучились. Во-первых, будете относиться к ней с б\'{о}льшим трепетом, во-вторых, где же я вам иначе сложные задачи найду.
\section{Сабж}
Сабж утверждает, что если есть множества A и B, причём A не мощнее B, а B не мощнее A, то A и B равномощны.

Так сразу и не поймёшь, где здесь вообще утверждение, поэтому можно попробовать переформулировать так, чтобы звучало страшнее. Пусть A и B --- множества, $\phi$ --- отображение, сопоставляющее каждому элементу A элемент из B (возможно, в B останутся ещё какие-то элементы), а $\psi$ --- отображение, сопоставляющее каждому элементу B элемент из A (теперь лишние элементы могут остаться уже в A). Теорема Кантора--Бернштейна утверждает, что в этом случае между A и B есть взаимно однозначное отображение (при котором никаких лишних элементов, конечно, уже не остаётся ни с какой стороны).

Поскольку вторая формулировка звучит внушительней, будем, доказывая теорему, оглядываться на неё. И вообще, нарисуем картинку.
%вставить картинку и доказать теорему
А теперь сразу задачи, без всяких пауз и разглагольствований.
\chapter*{Сразузадачи}
%задачи на сабж
\chapter{Последний день. Возведение в степень}
\section{Обозначение}
Обозначение, потому что на самом деле $A^B$ --- всего лишь обозначение для множества отображений из B в A, и единственное, что интересует нас --- это откуда это самое обозначение взялось и почему оно удобно.

Чтобы понять, можно, например, для начала посмотреть, что бывает, когда A и B --- конечные. В этом самом случае каждый элемент B переходит в какой-то из элементов A, то есть на каждый элемент B приходится по $|A|$ способов отображения, и мощность $A^B$ равна $|A|^|B|$, чего нам, конечно же, и хотелось бы (вспомним к тому же, что мощность --- основное, что нас интересует в множествах). Для бесконечных множеств это рассуждение тоже может более или менее работать, если забыть о строгости (чего вообще-то делать не стоит, но мы будем из вящей жалости к вам и ко мне). Поэтому если есть, например, множество $\mathbb N$, которое отображается в $\{0, 1\}$, элементов в этом множестве $2^{\mathbb N}$. А множество отображений из $\mathbb N$ в $\{0, 1\}$ мы, кстати, уже видели: это всевозможные последовательности счётной длины из нулей и единиц. Правда, когда мы беседовали о континууме, последовательности с хвостом из единиц мы не рассматривали, а теперь неожиданно рассматриваем, и чтобы понять, что $2^{\mathbb N} \sim \mathbb R$, нужно мимоходом доказать ещё одну маленькую лемму.

\begin{lemma}
Если A не менее чем счётно, а B не более чем счётно, то $A\cup B \sim A$.
\end{lemma}
\begin{lproof}
\emph{Рассмотрим любое счётное подмножество $C \subset A$. Мы знаем, ибо доказывали уже, что $C \cup B$ счётно, то есть $C \cup B \sim C$, поэтому $A=C\cup (A\setminus C) \sim B\cup (C \cup (A\setminus C)) = B \cup A$. Чего нам и хотелось.}
\end{lproof}

Вот так, осталось только доказать, что последовательностей с хвостом из единиц счётное число. Для этого обрубим у каждой последовательности хвост --- останется конечная последовательность --- и поймём, что если сопоставить этой последовательности вещественное число, как мы делали в главе про континуум, оно непременно будет рациональным, а рациональных чисел счётное множество. На этом всё, и $2^{\mathbb N} \sim \mathbb R$.

Для множеств верно, что $A^B \times A^C \sim A^{B+C}$; $(A \times B)^C \sim A^C \times B^C$. С точностью до изоморфизма можно оперировать со степенями множеств, как со степенями чисел.

\begin{exe}Проверить это, вспомнив, что вообще такое $A^B$.
\end{exe}
\chapter{Задачи на степени}
%задачи на степени
\end{document}



