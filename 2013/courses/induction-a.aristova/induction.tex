\documentclass[12pt]{article}
\usepackage{amssymb}
\usepackage[utf8]{inputenc}
\usepackage[russian]{babel}
\usepackage{amsthm}
\begin{document}
\theoremstyle{definition}
\newtheorem{zz}{Задача}
\newcommand{\problem}[1]{\begin{zz} $\ \ $ #1 \end{zz}}

\problem{ Докажите, что:
	\begin{enumerate}
	\item $1+2+3+...+n=\frac{n(n+1)}{2}$
	\item $1^2+2^2+3^2+...+n^2=\frac{n(n+1)(2n+1)}{6}$
	\item $\frac{1}{1\cdot 2}+\frac{1}{2\cdot 3}+\frac{1}{3\cdot 4}+...+\frac{1}{n\cdot (n+1)}=\frac{n}{n+1} $
	\end{enumerate}
}


\problem{
Докажите, что сумма углов n-угольника равна $180^{\circ}(n-2)$.
}

\problem{
Докажите, что если $q \ne 1$, то
$1+q+q^2+...+q^n=\frac{1-q^{n+1}}{1-q}$
}

\problem{
Докажите, что число слов из букв А и Б длиной n равно $2^n$.
}

\problem{
Докажите, что любую денежную сумму достоинством больше 7 рублей можно разменять монетами по 3 и 5 рублей.
}

\problem{
Найдите суммы:
\begin{enumerate}
\item $1+3+5+...+(2n-1)$
\item $1^2+3^2+5^2+...+(2n-1)^2$
\item $1^3+2^3+3^3+...n^3$\\ Подсказка: эта сумма всегда является квадратом некоторого целого числа. Чему равно это число?
\end{enumerate}
}
	
\problem{
Докажите, что всякие n квадратов можно разрезать на части так, чтобы из полученных частей можно было сложить новый квадрат.
}

\problem{
Докажите, что $(1+a)^n\ge 1+na$ при $a>-1$.
}
\problem{
Докажите, что части, на которые k прямых делят плоскость, можно раскрасить в два цвета так, чтобы никакие две части одного цвета не граничили по отрезку.
}

\problem{
Прямыми общего положения называется такой набор прямых, в котором никакие две прямые не параллельны и никакие три не проходят через одну точку. Найдите, на сколько частей делят плоскость n прямых общего положения.
}

\problem{
Докажите, что для натуральных n:
\begin{enumerate}
\item $2^n>n$
\item $2^n>n^2$ при $n>4$
\item $n!>2^n$ при $n>3$
\item существует k, такое, что $2^n>n^{1000}$ при $n>k$.
\end{enumerate}
}

\problem{В самолёт с N $(N\ge2)$ местами последовательно заходят N пассажиров. У каждого пассадира есть билет на определённое место. Первый пассажир - это сумасшедшая старушка, которая садится на произвольное место. Все остальные пассажиры действуют так: если их место свободно, то занимают его, а если нет - тогда любое случайное из незанятых. С какой вероятностью последнему пассажиру всё-так удастся сесть на своё место?
}

\problem{ Докажите, что:
\begin{enumerate}
\item $(2^{5n+3}+5^n\cdot 3^{n+2})\vdots 17 $
\item $(n^{2n-1}+1)\vdots (n+1) $
\end{enumerate}
}

\problem{ Докажите, что при любом n существует n-угольник с ровно тремя острыми углами.
}

\problem{ Докажите, что если число $a+\frac{1}{a}$ - целое, то тогда числа
\begin{enumerate}
\item $a^{2n}+\frac{1}{a^{2n}} $
\item $a^{n}+\frac{1}{a^n}$
\end{enumerate}
тоже целые.
}

\problem{
Докажите, что квадрат размером $2^n\times 2^n$ из которого вырезали одну
\begin{enumerate}
\item угловую
\item произвольную
\end{enumerate}
клетку можно разрезать на "уголки" из трёх клеток.
}

\problem{
Докажите, что число $111...1$ из $3^n$ единиц делится на $3^n$. (Подсказка: докажите, что число, запись которого содержит $3^n$ единиц делится на число из $3^{n-1}$ единиц.
}

\problem{ В скольких точках пересекаются n прямых общего положения?
}

\problem{ Проведено m отрезков, вершины которых лежат в вершинах правильного n-угольника. Докажите, что при $m\le n-2$ найдутся 2 вершины n-угольника, не соединённые ломаной из этих отрезков.
}

\problem{ Докажите, что $1+\frac{1}{\sqrt{2}}+...+\frac{1}{\sqrt{n}}>\sqrt{n}$ при $n\ge 2$.
}

\problem{ Шеренге новобранцев была дана команда "Налево!". Но поскольку новобранцы ещё не очень хорошо научились отличать право и лево, часть из них повернулась налево, а часть - направо. Дальше каждую секунду происходит следующее: если два солдата оказались лицом к лицу, каждый из них поворачивается кругом. Докажите, что этот процесс рано или поздно остановится.
}

\problem{
Докажите, что $\sqrt{6+\sqrt{6+\sqrt{6+\sqrt{6+\sqrt{6+\sqrt{6}}}}}}<3$.
}

\problem{
В колонию бактерий попал вирус. Каждую секунду каждый из вирусов уничтожает одну бактерию, удваиваясь при этом. Все оставшиеся в живых бактерии тоже удваиваются. Выживет ли колония бактерий?
}

\problem{Докажите, что при любом n:
\begin{enumerate}
\item $n^3+5n\ \ \vdots\ \ 6$
\item $n^3+9n^2+26n+24\ \ \vdots\ \ 6$
\item $7^{2n}-1\ \ \vdots\ \ 24$
\item $13^n+5\ \ \vdots \ \ 6$
\item $15^n+6\ \ \vdots\ \ 7$
\end{enumerate}
}

\problem{
Придя на встречу, её участники принялись пожимать друг другу руки. Докажите, что число тех, кто сделал нечётное количество рукопожатий, чётно.
}

\problem{
Докажите, что $\frac{1}{n+1}+\frac{1}{n+2}+...+\frac{1}{2n}>\frac{1}{2}$ при $n>1$.
}

\problem{
Один выпуклый многоугольник находится внутри другого. Докажите, что периметр первого многоугольника меньше периметра второго.
}

\problem{
На доске записано рациональное число $q=\frac{m}{n}$. По правилам игры его можно стереть и написать либо $q+1$, либо $\frac{1}{q}$. Докажите, что с помощью таких операций, начав с числа 1, можно получить любое положительное рациональное число. (Подсказка: попробуйте провести индукцию по n).
}

\problem{
Есть по одной гире весами в 1, 3, 9, 27, 81 и т.д. граммов. Докажите, что с их помощью можно уравновесить на двухчашечных весах любой груз, весящий целое число граммов, если можно класть гири на каждую из чашек.
}

\problem{
На доске написано слово длины n из букв А и Б. За один ход можно изменить одну из букв. Докажите, что за несколько ходов можно получить все $2^n$ возможных слов, при этом так, чтобы ни одно слово не встречалось в этой последовательности слов дважды. 
}

\problem{
(Задача о ханонойских башнях). Имеются три стержня. На первом из них лежит пирамида из n колец, при этом если одно кольцо лежит над другим, то оно меньшего размера. Разрешено перекладывать кольца на другие стержни так, чтобы это свойство сохранялось. Докажите, что:
\begin{enumerate}
\item для любого n можно переложить пирамиду с первого стержня на третий;
\item для такого перекладывания достаточно $2^n-1$ действий;
\item меньшим числом действий обойтись нельзя.
\end{enumerate}


}

\end{document}
