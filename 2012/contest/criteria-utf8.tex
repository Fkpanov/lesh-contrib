\documentclass[10pt, a4paper, oneside, fleqn]{article}
\usepackage{polyglossia}
\usepackage{unicode-math}
\usepackage{fontspec}
\usepackage{hyperref}
\usepackage{anysize}
\usepackage{makeidx}

% поля должны быть такими, чтобы на них можно было делать заметки
\usepackage[left=25mm,right=20mm,top=15mm,bottom=20mm]{geometry} % поля страницы

\setdefaultlanguage[spelling=modern]{russian} %% Адекватные переносы.
\setotherlanguage{english}

\setmainfont{DejaVu Serif} %% задаёт основной шрифт документа
\setsansfont{DejaVu Sans} %% задаёт шрифт без засечек
\setmonofont{DejaVu Sans Mono} %% задаёт моноширинный шрифт

\setmathfont{xits-math.otf}

% the beginning of DMVN package
\newcommand\hr[1]{\left({#1}\right)}
\newcommand\т{~--- }
\def\le{\leqslant}

\title{Критерии оценки задач\\
Заочной межрегиональной\\
физической олимпиады 2012---2013}
\author{Арсений Ларцев, Алёна Жигулина, Александр Трусевич\\
\url{http://fizlesh.ru/contest}}

\begin{document}
\addfontfeatures{Mapping=tex-text} %% Касаемо тире.

\maketitle

\section{Теоретические задачи}

\subsection{Воздушные шарики}

\begin{itemize}
\item $1$~балл за правильный \emph{ответ}
\item $3$~балла за расчёт \emph{силы Архимеда} либо $1$~балл за достаточно развёрнутое \emph{качественное обоснование}.
Например, <<водород имеет меньшую плотность>> считалось достаточно развёрнутым, а <<водород легче>>\т нет
\item $1$ балл за упоминание любых дополнительных факторов, влияющих на результат
\item $1$ балл за рисунок, на котором изображены \emph{действующие силы}
\end{itemize}

\textbf{Дополнительные баллы} (по $1$ баллу за каждый из факторов):
\begin{itemize}
\item растяжимая оболочка
\item диффузия газа сквозь оболочку
\item возможное возгорание водорода (взрывоопасность)
\item тепловое расширение при изменении температуры
\end{itemize}

\subsection{Отскок}

\begin{itemize}
\item $5$ баллов за верное решение
\item $\le 2$ за красивую идею
\item $+1$ за оценку результата, \emph{насколько выше} подпрыгнет
\item $+1$ за объяснение решения (достаточно подробное объяснение того, как работает предложенная конструкция)
\item $+1$ за описание реального эксперимента
\item $+1$ за несколько вариантов
\item $-1$ за отсутствие ссылок на законы, неочевидное следствие
\item $\le 2$ за решение \emph{не той} задачи (неправильно понятое условие)
\end{itemize}

\subsection{Тонкий слой}

ГДЕ КРИТЕРИЙ-ТО?

\subsection{Теплоёмкость}

\begin{itemize}
\item Если решение предполагало нагрев воды до нескольких сотен градусов без испарения
без указания на то, что для реализации этого нужно \emph{огромное давление}, то оно оценивалось в $0$~баллов!

\item В $9$---$10$ баллов оценивалось решение, в котором площадь под графиком посчитана разбиением
на достаточное большое число прямоугольников, трапеций (или пикселей) и оценена \emph{погрешность} этого
измерения.

\item Если площадь под графиком подсчитана разбиением на меньшее число прямоугольников или трапеций,
которое не гарантирует достаточной точности, то решение оценивалось в $5$---$7$~баллов.

\item Решение, в котором вся площадь под графиком приближалась одной трапецией, оценивалось в $3$---$4$ балла.

\item Решение, предполагавшее теплоёмкость постоянной и равной оценённому <<на глаз>> среднему значению,\т в~$2$---$3$ балла.

\item Любая попытка к решению, не увенчавшаяся успехом, оценивалась в $1$~балл.

\item $1$~дополнительный балл ставился за учёт возможности охлаждения и/или испарения воды.
\end{itemize}

\subsection{У кого трава зеленее}

\begin{itemize}
\item Если предлагался вариант, что на соседском газоне хуже видно землю промеж травы, потому что он
\emph{виден под более острым углом} к~поверхности земли, то решение оценивалось в~ $5$~баллов.
\item Если было утверждение, что на соседском газоне хуже видно землю,
но \emph{не было} указано, что это происходит из-за разных углов зрения,\т $4$~балла.
\item Если в решении упоминалась разница в углах зрения, но \emph{не было} вывода о том,
что на соседском газоне хуже видно землю,\т $3$ балла.
\item Если лишь упоминались причины, почему соседский газон более зелёный
сам по себе (\emph{другой сорт травы, больше травы, лучше освещён, богатый полив}), то ставилось $1$---$2$ балла.
$2$ балла ставилось, если в работе упоминались какие-либо \emph{физические явления}.
\item До $2$ дополнительных баллов ставилось за поясняющий \emph{рисунок}.
\end{itemize}

\section{Экспериментальные задачи}

\subsection{Поверхностное натяжение}

В работе оценивалось (по $1$~баллу за каждый пункт):
\begin{itemize}
\item корректно работающий метод измерения
\item описание схемы установки
\item описание принципа действия установки
\item предоставление всех экспериментальных данных
\item представление конечного результата
\item теоретическое объяснение эффекта
\item теоретическое описание работы установки
\item подробный чертёж или фотография установки
\item подробное представление экспериментальных данных
\item расчёт погрешностей
\item правильный порядок величины полученного результата
\end{itemize}

\subsection{Пинг-понг}

\begin{itemize}
\item $3$ балла: решение с использованием \emph{закона сохранения энергии} или \emph{закона Гука}
\item $1$ балл за выполнение \emph{эксперимента}
\item $\le 2$ баллов: \emph{картинка} установки (в зависимости от качества)
\item $1$ балл: в решении зафиксированы \emph{экспериментально полученные данные}
\end{itemize}

Работа, не набравшая баллов по приведённым выше критериям, но содержащая любые \emph{разумные рассуждения}, оценивалась в $1$~балл.

\subsection{Мираж}

\begin{itemize}
\item $1$~балл за описание явления \emph{миража}
\item $3$~балла за предложенный опыт, если он корректен, то есть использует нагревание воздуха для \emph{искривления хода лучей}
\item $1$~балл за объяснение опыта (неважно, корректного или нет).
\item $3$~балла за выполнение опыта, если он корректен (см. выше), либо $2$~балла, если преломление лучей
не было связано с нагреванием воздуха
\end{itemize}

\subsection{Женя и карандаш}

\begin{itemize}
\item $3$~балла за \emph{правильную идею} (моток проволоки, свисающий под стол, либо юла).
Однако, идея с юлой оценивалась только в $2$~балла, если не было явно упомянуто \emph{вращение}

\item $1$ балл за объяснение, почему карандаш будет стоять на острие

\item $1$ балл, если предложено \emph{более одного способа} (даже если все предложенные альтернативные способы не работают)

\item До $3$~баллов за какие-либо \emph{расчёты} (например, времени, в течение которого карандаш простоит на острие)
и \emph{сравнение эффективности} разных способов

\item $1$ балл за особо \emph{оригинальные идеи} (пусть даже не работающие на практике)
\end{itemize}

\subsection{Падающая башня}

\begin{itemize}
\item $1$ балл за любое упоминание какого-либо \emph{эксперимента}
\item $\le 4$ баллов за \emph{хорошо поставленный эксперимент}.
В~частности, оценивались \emph{изображения} конструкции, \emph{фотографии} процесса,
идея измерить \emph{отношение двух частей башни}
\item $1$ балл за любую попытку объяснения наблюдаемого явления
\item $\le 5$ баллов за физически корректное объяснение
\end{itemize}
Максимальная оценка за эту задачу составила 5 баллов.

\end{document}